\documentclass{subfiles}
\begin{document}
    Let \(x\) be a positive Integer, 
        its delta code (\(\delta(x)\)) is a binary sequence composed by:
        \begin{itemize}
            \item the delta code of \(\abs{B(x)}\)
            \item the binary representation of \(x\) excluding the most significant bit
        \end{itemize}

    \begin{example*}
        Let \(x = 11\), then its delta code is:
            \[
                \delta(x) = 001 00 011
            \]
        where \(00100\) is the gamma encoding of \(\abs{B(x)}\) and 
            \(011\) the binary encoding of \(x\), most significant bit excluded.
    \end{example*}
    
    As before, the decoding is trivial, so we skip it.
    Once can prove that the code uses at most \(1 + \log x + 2 \log \log x\) bits.
    Also, and most importantly,
        gamma codes are reasonable when \(p(x) = \tfrac{1}{2x (\log x)^{2}}\).
\end{document}
