\documentclass{subfiles}
\begin{document}
    FASTA (or FAST-All, see \cite{lipman1985} for major details) is a heuristic 
        local alignment algorithm; that is, it finds the best possible alignment,
        in the heuristic sense, in a relatively short amount of time. 

    Assume we have a sequence \(s\) and we want to find the most similar sequence 
        in a database\footnotemark. We proceed by pre-computing an index of positions as follow:
        fix an integer \(k\) and compute all factors of \(s\) of lenght \(k\);
        for each of these factors we store the possion where this occur in the database
        sequences. For instance, let \(s = ACGGTYFF\) and let 
        \[\begin{aligned}
            w_{1} = ACCGFFNFG \\
            w_{2} = DFCGFFNFG \\
            w_{3} = LACGFFNFG \\
        \end{aligned}\]
        be the database sequences. Fix \(k = 2\), the factors of \(s\) then are: 
        \(AC, CG, GG\), \(GY, TY, YF, FF\). Considering for example the factor \(AC\),
        we have that it occurs at position 0 in \(w_{1}\), at position 1 in \(w_{3}\)
        and does not occur in \(w_{2}\). We do the same for each other factor.

    Once the index is computed, we try to identify the diagonals, as done in the SW-algorithm;
        note that in this case the diagonals is given by the diffence of the positions in the factors of 
        \(s\) and their occurence in each of the \(w_{i}\).
        We then recalculate the score of those regions of the longest segments by means of 
        a substitution matrix, e.g. BLOSUM-62. For each of the databases sequences, 
        consider the ten regions with highest score and compute their sum \emph{Init1}.
        Sort the sequences by this value. It could happen that some of the initial regions
        can be merged in the same alignment if gaps are added. If this happens,
        we try adding such gaps and then sum the score of these region minus some penalty for the gaps.
        Call such score \emph{InitN}. Lastly, consider the sequences with the highest 
        InitN, apply a variant of SW to them.

    It's easy to see that FASTA is really fast, up to 10 times faster then SW;
        but some severe drowbacks are also evident: the heuristic nature of the algorithm 
        makes it unsuitable for high precision alignments, also, the efficiency depends on the value 
        of \(k\), the lower this is, the faster is the algorithm with the caveat that in the first phase 
        an indenty criterion is used.


    \footnotetext{By database we refer to biological databases, 
        which we describe in \emph{Section:4.3}.}
\end{document}
