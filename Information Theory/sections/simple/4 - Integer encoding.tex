\documentclass{subfiles}
\begin{document}\label{Sec:4}
    What happens when the source to encode uses positive integers as symbols?
        How can we find a universal representaion? 
        Meaning, how can we find a code that is prefix free,
        and whose \gls{acl} is \bigO{\ln x}, for any \(x\).

    Can we still use the codes seen so far, or do we need new ones?
        The answer is we can do both depending on the case.
        For instance, if we know the distribution of such integers and the range is
        relatively small, we can still use Huffman. 
        If not, we need to use some new codes: those we are about to discuss.

    \begin{remark*}
        We can relax the constrain on the positiveness, 
            by mapping any positive integers \(x \text{ to } 2x + 1\) and 
            any negative integers \(y \text{ to} -2y\).
    \end{remark*}

    \subsection{Simple approaches}
    \subfile{../sub/4.1 - IE simple approaches}

    \subsection{Elias codes}
    \subfile{../sub/4.2 - Elias codes}

    \subsection{Fibonacci code}
    \subfile{../sub/4.3 - Fibonacci code}

    \subsection{Exercises}
    \subfile{../sub/4.4 - IE exercises}
\end{document}
