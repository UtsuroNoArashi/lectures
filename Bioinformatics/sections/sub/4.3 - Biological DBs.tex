\documentclass{subfiles}
\begin{document}\label{Sec:4.3}
    So far it has been discussed on how we process DNA/RNA sequences;
        a question arise naturally: how do we store such sequences, along
        with all the indormations we derived? 
        The obvious solution is by means of databases. 
        
    The not so trivial problem now is how to keep the data coherent across all 
        these databases. We note that these are distributed all over the word 
        since the amount of data to be stored doesn't allow a localized solution.
        As in the case of classic databases, the best approach is that of 
        updating each database once new inforamtions are available.
        For this reason an agreement was made across all international institutes.

    Focusing on the the structure of these databases, each usually focuses on 
        a singular biological element (eg. only nucleotide, only amino acids, etc).
        Note that the informations stored in different databases is the same,
        the only diffence are in how these are stored (flat-files, fasta-format, etc).
        
    For the sake of completeness, a nucleotide centerd databases is often refferd to as 
        a primary database, as it contains the essential informations of a given sequence;
        whilest amino-acid centerd ones are refferd to as secondary databases.

\end{document}
