\documentclass{subfiles}
\begin{document}
    As stated multiple times throughout these notes, 
        a gene mutate as consequence of nucleotide substitutions, deletions, etc.
        We'd like to note that not all variations of a gene are significant,
        as it could just concern a really small portion of the species.
        On the other hand if a mutation iterests a large of the population, 
        where by large, in practice, we consider anything greater than the 1\%, it makes sense to study 
        such a gene.

    Now, the mutation itself may occur either due to \emph{natural selection}
        or \emph{neutral genetic drift}. The former refers to the capacity for reproduction
        of genetically distinc individual within the population; the latter on the other hand
        is a purely random process. On the practical side there are two distinc approaches 
        to study/estimate substitution: the deterministic one, based purely on observations,
        and the stochastic one, which make use of time-dependent variables.
        Despite we don't describe any of them, since several exist in the literature,
        we consider stochastical approaches. In some cases we refer to the \emph{molecular
        clock hypothesis}, that is the idea for which the number of substitutions 
        depends on the time elapsed after their divergence.
\end{document}
