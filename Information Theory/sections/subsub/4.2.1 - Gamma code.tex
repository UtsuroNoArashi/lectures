\documentclass{subfiles}
\begin{document}
    Let \(x\) be a positve Integer.
        Its gamma code (\(\gamma(x)\)) is a binary sequence composed by:
        \begin{itemize}
            \item the unary encoding of \(\Abs{B(x)}\)
            \item the binary representaion of \(x\) minus the most significant bit.
        \end{itemize}

    \begin{example*}
        If \(x = 11\), then its gamma encoding is:
            \[
                \gamma(x) = 0001011
            \]
        where \(0001\) is the unary encoding of \(\Abs{B(x)}\) and \(011\)
        the binary representaion of \(x\) minus the most significant bit.
    \end{example*}

    The decoding is trivial, so we skip it.
    It's easy to prove that with this kind of code, we need at most \(2 \Floor{\log x} + 1\)
    bits. Also, most importantly, this encoding is reasonable when \(p(x) = 2x^{-2}\).
\end{document}
