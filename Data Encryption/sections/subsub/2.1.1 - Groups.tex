\documentclass{subfiles}
\begin{document}
    \begin{definition*}[Group]
        Let \(G\) be a set. Assume \(\cdot : G \to G\) to be some function. 
            We say that the pair \((G, \cdot)\), often simply as \(G\),
            is a group if the following properties hold: 
            \begin{itemize}
                \item \(\forall a, b, c \in G, a \cdot (b \cdot c) = (a \cdot b) \cdot c\);
                \item \(\exists e \in G \ \forall g \in G \text{ s.t. } g \cdot e = e \cdot g = g\);
                \item \(\forall g \in G \ \exists g^{-1} \in G \text{ s.t. } g \cdot g^{-1} = g^{-1} \cdot g = e\).
            \end{itemize}

        If it also holds that \(\forall a, b \in G, a \cdot b = b \cdot a\), 
            we say that \(G\) is an abelian (or commutative) group.
    \end{definition*}

    We consider finite groups: that is, group with finitely many elements.
        Precisely, we are interested on the following class of groups.

    \begin{definition*}[Cyclic group]
        Let \(G\) be a group, and let \(g \in G\). Denote by \(g^{k}\) the application 
            of \(\cdot\) on \(g, k\) times. Denote by \(\gen{g}\) the set of all 
            powers of \(g\). We say that \(G\) is cyclic if \(\gen{g} = G\).
    \end{definition*}

    \begin{definition*}[Subgroup]
        Let \(G\) be a group, and let \(H \subseteq G\). 
            We say that \(H\) is a subgroup of \(G\), if \(H\) is a group under 
            the operation inherited from \(G\).
    \end{definition*}

    \begin{theorem*}[Langrange's]
        Let \(G\) be a group, and let \(H\) be one of its subgroups.
            Let \(\ord{G}, \ord{H}\) denote the order (the number of elements) of 
            \(G \text{ and } H\), respectively. Then, it holds
            \[
                \divides{\ord{H}}{\ord{G}}.
            \]
    \end{theorem*}

    When the group in question is cyclic, Langrange's theorem can be states as follow:
        let \(g \in G\) and denote by \(\ord{g}\) the smallest integer \(n\) such that \(g^{n} = 1\),
        then, it holds \(\divides{\ord{g}}{\ord{G}}\).

    We will implicitely use Langrange's theorem when we discuss RSA and ElGamal 
        algorithms.

\end{document}
