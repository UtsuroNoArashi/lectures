\documentclass{subfiles}
\begin{document}
    From the discussion of the BWT, a question might arise naturally: 
        how much does the compression improve by preprocessing the string using the BWT?
        The question to this question can be found in \cite{manzini2001}.
        Here Manzini analyzes the compression of BWT-based algorithm,
        in terms of their \emph{empirical entropy\footnotemark[4]}.
        Even though the paper referes just to the 0-th order empirical entropy,
        defined as follow,
        \[
            \entropy{s}[0] = - \sum_{c \in \Sigma}
                \frac{n_{c}}{n} \log \frac{n_{c}}{n}  
        \]
        where \(n_{c}\) is the number of occurences of \(c \text{ in } s\);
        one could generalize the idea to a k-th order empirical entropy as follow
        \begin{equation}\label{Eq:2}
            \entropy{s}[k] = \frac{1}{\abs[s]} 
                \sum_{w \in \Sigma^{k}} \abs[w_{s}] \entropy{w_{s}}[0]
        \end{equation}
        where \(w_{s}\) is the concatenation of symbols that follow \(w \text{ in } s\).
    \footnotetext[4]{Due to the way it's defined, see \eqref{Eq:2},
        empirical entropy is perfect for a worst-case analysis.}

        In the paper Manzini provides also the following results.
        \begin{theorem*}[Manzini]
            For any string \(s\) over some alphabet \(\Sigma\) and for any \(k \ge 0\),
            it holds that 
            \[
                BW_{0} = \le 8 \abs[s] \entropy{s}[k] +
                    \left\lparen \mu + \frac{2}{25} \right\rparen \abs[s] + 
                    h^{k} (2h \log h + 9)
            \]
            where \(h = \abs[\Sigma] \text{ and } \mu = 1\).
        \end{theorem*}
        Let us point out that \(BW_{0} = Order_{0} + MTF + BWT\), 
        where \(Order_{0}\) is some 0-order compressor.
        \begin{theorem*}
            For any string \(s\) over some alphabet \(\Sigma\) and for any \(k \ge 0\),
            there exists some constant \(g_{k}\), such that it holds
            \[
                BW_{0} + RL \le (5 + 3\mu)\abs[s] H_{k}^{*}(s) + g_{k}
            \]
            where \(RL\) is the run length encoding of the string.
        \end{theorem*}

        To conclude this section, 
            let us observe that the bottleneck of the BWT is
            given by the sorting of the cyclic rotations.
            We should show, in the following section, 
            that the problem can be reduced to the sorting of suffixes.

        \subsubsection{Efficient computation of the BWT}
        \subfile{../subsub/5.2.1 - Efficient BWT computation}
\end{document}
