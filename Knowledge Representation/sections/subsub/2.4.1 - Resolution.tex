\documentclass{subfiles}
\begin{document}\label{Sec:2.4.1}
    Resolution is a theorem-proving technique based on refutation, i.e., 
        we try to get a contradiction. To work properly, 
        we have to write any given formula in a clause form, that is,
        we have to write the formula as disjunction of literals.
        Therefore, each disjunction in a CNF is a clause. 
        To ease the notation we represent clauses as sets, where the clause \(\set{ }\) 
        means false.

    \begin{example*}
        Consider \(F = (\lnot p \lor q) \land (q \lor r)\), with our notation 
        we write \(F\) as the set of sets \(\set{ \set{\lnot p, q}, \set{q, r}}\).
    \end{example*}

    But how do we proceed? Essentially, given a formula in clause form 
        we repeatedly apply the \emph{resolution rule}; i.e., 
        given two clauses \(C_{1} \cup \set{A}\) and \(C_{2} \cup \set{\lnot A}\) 
        we get \(C_{1} \cup C_{2}\). That is, we look for pairs of clauses
        with two opposing literals and we merge them. 
        We proceed this way until either we have no more clauses satisfing the above, 
        or we get the empty clause, in which case we get the wanted contradiction.
\end{document}
