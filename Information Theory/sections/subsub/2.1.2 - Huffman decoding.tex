\documentclass{subfiles}
\begin{document}
    Let's assume we recieve a text encoded using Huffman, how do we decode it?
    We shall remark that to decode the text we need to know either the Huffman tree, 
        or the source probabilities, which represent in both cases an overhead
        to the bare encoding.
    Assuming that such overhead is know, the decoding is immediate.
    In fact, we read the recieved text and traverse the tree accordingly 
    until reaching a leaf. Repeating the process for the whole text.

    \begin{example*}
        Let \(\omega = 00101111100001\) be a text encoded using Huffman.
        Let's assume that the tree used is the one in \emph{Figure \ref{Fig:2.b}}.
        Starting from the root, we read a zero as first symbol of the encoded text,
        thus we consider the left subtree. 
        Since we are not at a leaf we keep reading the encoded text.
        We read another zero, so once again we consider the left subtree.
        Since we reached a leaf we stop and decode \(00\) as \(a\). 
        We move back to the root and repeat the process. 
        After few steps we decode \(\omega = abedac\).
    \end{example*}
\end{document}
