\documentclass{subfiles}
\begin{document}
    The idea is to divide the text in two parts: 
        a \emph{search buffer} -- a fixed size portion of the text already processed --
        and a \emph{lookahead buffer} -- the portion of text not yet seen;
        which are identified by two pointers \(i, j\) as shown in \emph{Figure \ref{Fig:7}}.
        \subfile{../../extra/TikZ/Figure 7 - LZ text splitting}
        
        Basically, we look for the longest existing pattern in the search buffer,
        that is prefix of the lookahead buffer.
    The compression produces a set of triplets \(<f, l, c>\) where 
        \(f\) is the distance between \(i \text{ and } j\),
        \(l\) is the length of the found prefix and \(c\) is the symbol after the prefix.

        \begin{example*}
            Let \(S = aabbaaabaca\). Let's assume we are in the situation below.
                \subfile{../../extra/TikZ/Figure * - Example of LZ77 compression}   
            
            \noindent Since the search buffer contains a patter that is prefix of the lookahead,
                we return the triplet \(\ <5, 3, a>\).
        \end{example*}

    The decompression is easy, and, as stated previously, 
        doesn't require the dictionary to be transmitted.
        In fact, after reciveing a triplet, 
        we look \(f\) characters deep into the search buffer (that is initially empty),
        copy \(l\) symbols into the lookahead buffer 
            (which at this point represent the text being decoded at each step) and 
            concatenate \(c\) to it.
        If needed, we move the search buffer.

        \begin{example*}
            Let us consider the triplet \(<5, 3, a>\) of the previous example.
            Again, we assume we are in a situation like the one below. 
                \subfile{../../extra/TikZ/Figure * - Example of LZ77 decompression}
            
            \noindent We look \(5\) characters deep into the search buffer, 
                copy \(3\) symbols (abb) into the lookahead buffer and add an \(a\) at the end.
        \end{example*}

    \subsubsection{LZss: a variation to LZ77}
    \subfile{../subsub/6.2.1 - LZss a variantion to LZ77.tex}
\end{document}
