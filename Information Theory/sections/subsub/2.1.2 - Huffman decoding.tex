\documentclass{subfiles}
\begin{document}
    Let's assume we receive a text encoded using Huffman. How can we decode it?
    We remark that decoding requires knowledge of either the Huffman tree 
        or the source probabilities, which in both cases constitute an overhead
        relative to the bare encoding.
    Assuming this overhead is known, decoding is straightforward.
    Specifically, one reads the encoded text and traverses the tree accordingly 
    until reaching a leaf. This process is repeated until the entire text is decoded.

    \begin{example*}
        Let \(\omega = 00101111100001\) be a text encoded using Huffman.
        Assume the tree used is the \textbf{(\subref{Fig:1.b})} one of \emph{Figure \ref{Fig:1}}.
        Starting from the root, the first symbol in the encoded text is zero,
        so the left subtree is considered. 
        Since the current node is not a leaf, reading continues.
        Reading another zero, we again follow the left subtree.
        Upon reaching a leaf, we decode \(00\) as \(a\). 
        The traversal then returns to the root, and the process is repeated. 
        After a few iterations, the decoded text is \(\omega = abedac\).
    \end{example*}
\end{document}
