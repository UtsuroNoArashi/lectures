\documentclass{subfiles}
\begin{document}
    Since the BWT is a transform, we want it to be reversible.
    As obvious, the BWT is indeed reversible, 
        and actually it has two distinct, yet equivalent, inverse transform:
        the FL-mapping and the FL-mapping.

    Let us use the BWT of the previous example as input to the inverse transforms.

    \paragraph*{FL-mapping}
    Let \(L = BWT(aleph) \text{ and } I = 0\) the index of the original string.
    We construct \(F\) by lexicographically sorting \(L\).
    Let's define the permutation that maps the symbols in \(F\) 
    to those in \(L\) as 
    \[
        \tau = \begin{pmatrix}
            0 & 1 & 2 & 3 & 4 \\ 
            3 & 4 & 0 & 1 & 2 \\
        \end{pmatrix}\text{.}
    \]
    We recover \(\omega\) by computing 
        \(\omega[i] = F[\tau^{i}[I]]\), for \(i = 0, 1, \ldots, 4\).
        In fact, we have: \(\omega[0] = F[I] = a, \omega[1] = F[\tau[0]] = F[3] = l\),
        and so on.

    \paragraph*{LF-mapping}
    Let \(L = BWT(aleph) \text{ and } I = \) the index of the original string.
    We construct \(F\) by lexicographically sorting \(L\). 
    Let's define the permutation that maps the symbols in \(L\) 
    to those in \(F\) as 
    \[
        \sigma = \begin{pmatrix}
            3 & 4 & 0 & 1 & 2 \\
            0 & 1 & 2 & 3 & 4 \\ 
        \end{pmatrix}\text{.}
    \]
    We recover \(\omega\) by computing 
        \(\omega[n - 1 - i] = L[\sigma^{i}[I]]\), for \(i = 0, 1, \ldots, 4\).
        In fact, we have: \(\omega[4] = L[I] = h, \omega[3] = L[\sigma[0]] = L[2] = p\) 
        and so on.
\end{document}
