\documentclass{subfiles}
\begin{document}
    \begin{definition*}[Group]
        Let \(G\) be a set. Assume \(\cdot : G \to G\) to be some function. 
            We say that the pair \((G, \cdot)\), often simply as \(G\),
            is a group if the following properties hold: 
            \begin{itemize}
                \item \(\forall a, b, c \in G, a \cdot (b \cdot c) = (a \cdot b) \cdot c\);
                \item \(\exists e \in G \ \forall g \in G \text{ s.t. } g \cdot e = e \cdot g = g\);
                \item \(\forall g \in G \ \exists g^{-1} \in G \text{ s.t. } g \cdot g^{-1} = g^{-1} \cdot g = e\).
            \end{itemize}

        If it also holds that \(\forall a, b \in G, a \cdot b = b \cdot a\), 
            we say that \(G\) is an abelian (or commutative) group.
    \end{definition*}

    We consider finite groups: that is, group with finitely many elements.
        Precisely, we are interested on the following class of groups.

    \begin{definition*}[Cyclic group]
        Let \(G\) be a group, and let \(g \in G\). Denote by \(g^{k}\) the application 
            of \(\cdot\) on \(g, k\) times. Denote by \(\gen{g}\) the set of all 
            powers of \(g\). We say that \(G\) is cyclic if \(\gen{g} = G\).
    \end{definition*}
\end{document}
