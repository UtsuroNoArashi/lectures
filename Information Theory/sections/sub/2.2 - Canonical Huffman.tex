\documentclass{subfiles}
\begin{document}
    Let us consider some limitations of the Huffman encoding discussed so far.
    One of the main problems, 
        is the fact that the encoder has to transmit alongside the text,
        the tree used for it: a not so trivial overhead.
    Additionally, 
        the decoding phase is slow due to the necessity to traverse the tree for each symbol.

    Let us observe that there exist codes, that are compact,
        that are not produced by the Huffman algorithm.
        We now introduce the so called \emph{Canonical Huffman encoding},
        which is a variant of Huffman, 
        that solves the issues above by requiring the encoder to transmit the lengthts of the codewords.

    Such new encoding, comes in handy when the source alphabet is large, 
        and a fast decoding is mandatory.

    \subsubsection{The encoding}
    \subfile{../subsub/2.2.1 - Canonical Huffman encoding}

    \subsubsection{The decoding}
    \subfile{../subsub/2.2.2 - Canonical Huffman decoding}

    \subsubsection{Exercises}
    \subfile{../subsub/2.2.3 - Canonical Huffman exercises}
\end{document}
