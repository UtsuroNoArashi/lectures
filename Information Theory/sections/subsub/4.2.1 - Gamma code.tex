\documentclass{subfiles}
\begin{document}
    Let \(x\) be a positive integer. Its gamma code, denoted 
    \(\gamma(x)\), is a binary sequence composed of the following elements:
    \begin{itemize}
        \item the unary encoding of \(\abs{B(x)}\), and
        \item the binary representation of \(x\) with the most significant bit removed.
    \end{itemize}
    Decoding is straightforward: count the nomber of zeros up to the first 1, 
        say they are \(k\), treat the next \(k + 1\) bits (1 included) as the integer \(x\).

    \begin{example*}
        For \(x = 11\), the gamma encoding is
        \[
            \gamma(x) = 0001011,
        \]
        where \(0001\) corresponds to the unary encoding of \(\abs{B(x)}\) 
        and \(011\) represents the binary representation of \(x\) with the 
        most significant bit removed.
    \end{example*}

    It can be shown that this 
    encoding requires at most \(2 \lfloor \log x \rfloor + 1\) bits. Most 
    importantly, this code is particularly efficient when the probability 
    distribution is \(p(x) = 2 x^{-2}\).
\end{document}
