\documentclass{subfiles}
\begin{document}
    We have seen that proteins are residues of amino acids, thus, to identify each proteins
        we can consider the amino acids in its primiry structure. This is exactly the role of DNA.
        Precisesly, each triplet of nucleotide (codone) encodes an amino acid.
        Let us observe that 4 nucleotides should produce 64 distinct amino acids,
        but as we have said, nature only provides us only 20. We can conclude that some codones identify the same acid.
        Additionally, some codones (UAG, UAA, UGA) do not encode any amino acid, 
        rather they are used to signal the end of a gene. 

    The process that leads to proteins synthesis begins with a phase know as \emph{transcription}.
        In this phase, a codone AUG identifies the begin of a gene which is copied onto an RNA molecule.
        We obtain this way the messanger RNA, or mRNA. Now, the process just described works only for 
        procaryotes; for eucaryotes things are a bit more complex. Essentially,
        eucaryotes genes are composed of aternating parts: the \emph{introns} and the \emph{exons}.
        We care only of exons. To solve this issue, once we get the mRNA the parts corresponding to introns are removed.

    Synthesis is done in a special cellular structure: the ribosome. 
        Ribosomes are composed by proteins and a new type of RNA, the ribosomal RNA.
        Here mRNA is read sequentially and a special type of RNA, the tRNA,
        carries the amino acid associated with the codone being read.

    To summarize: within DNA we produce copies of genes, 
        these are then transcibed via RNA and within ribosomes we synthesize the proteins.

    What we just described assumes no error occurs during DNA replication.
        As easily understood this is rarely the case. In fact, if errors did not occure,
        evolution would impossible. More in general, when dealing with DNA sequences we have to consider 
        mutations, i.e, nucleotide X becomes nucleotide Y, and/or gaps, 
        some nucleotide could get added/removed from the original sequence.
\end{document}
