\documentclass{subfiles}
\begin{document}
    Let us consider again the sentece: 
    \begin{quote*}
        If it rains, we always take the umbrella. Yesterday we took the umbrella,
        so yesterday it rained.
    \end{quote*}
    Is it necessary true? What if we took the umbrella to rapair it?

    This kind of reasoning is called deduction; that is, 
        deduction is the process that from a ``rule'' (the conditional) derives a conclusion.

    The above is an example of what we call \emph{affirming the consequent}; 
        that is, we deduce that A is true knowing that B is true, 
        when there exist a conditional statements involving A and B.
        As its counterpart we have the \emph{denying the antecedent} fallacy,
        in which we deduce the falsity of B by knowing with certainty that A is not true.

    Given a conditional sentece, the two correct ways to apply deduction are called 
    \begin{itemize}
        \item \foreignlanguage{latin}{modus ponens}: we know that A is true and deduce that B is also true. 
        \item \foreignlanguage{latin}{mudus tollens}: we know that B is false and deduce that A is also false.
    \end{itemize}
\end{document}
