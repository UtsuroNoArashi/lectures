\documentclass{subfiles}
\begin{document}
    SEQUITUR, introduced in \cite{witten1997} by Nevill-Manning et al., 
        later improved by SEQUENTIAL (\cite{yang2000}),
        is  one of the first algorithms that approaches the SGP.

    The algorithm is very simple: read the input string left-to-right,
        keep adding the symbols read to the grammar until a digram, is found twice.
        Replace the duplicate with a new non-terminal symbol.
        Repeat the procedure until the whole string is read. 

    By applying the above procedure, it may happen that a non-terminal symbol is 
        used only once; replace it with its definition.

    In other words, SEQUITUR is based on the digram uniqueness (the digram substitution describe above)
        and the rule utility, i.e. we define a new production if and only if this 
        would be used more then once.

    \begin{example*}
        Let's assume that our input is the string \(\omega = abbabbc\). 
        Applying the algorithm, We read the sequence \(abba\) without encountering
        any duplicated digram. Upon reading an additional \(b\), we note that the
        digram \(ab\) appear twice, hence, we proceede to update the grammar.
        Therefore, we have 
        \[\begin{aligned}
            S & \to AbA \\ 
            A & \to ab
        \end{aligned}\]
        We keep on reading and encounter a \(b\), which leads in the duplicate of 
        the digram \(Ab\). We update the grammar once again and get 
        \[\begin{aligned}
            S & \to BB \\
            A & \to ab \\ 
            B & \to Ab 
        \end{aligned}\]
        Finally, we read the \(c\) reaching the end of the string. 
        But we note that now \(A\) is used once, hence we update the grammar as 
        \[\begin{aligned}
            S & \to BBc \\ 
            B & \to abb \\
        \end{aligned}\]
    \end{example*}
\end{document}
