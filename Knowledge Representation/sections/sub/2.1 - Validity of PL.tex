\documentclass{subfiles}
\begin{document}
    Let \(P = \set{p_{1}, \ldots, p_{n}}\) be a set of atomic formulas, 
        let \(\Fam(P)\) be the set of all formulas that can be derived 
        from atomic propositions in \(P\).
    
    We say that a function \( A : P \to \set{\mathbf{0}, \mathbf{1}}\) is an assignment of \(P\);
        that is, \(A\) is an assignment of truth values to each atomic formula in \(P\).
        The extension of \(A\) to \(\Fam(P)\) is immediate.
        Given \(F \in \Fam(P)\) and \(A\) an assignment of \(P\), 
        we say that \(A\) is a model of \(F\) if \(A(F) = \mathbf{1}\);
        we write \(A \models F\) to denote this concept. 
        If \(A(F) = \mathbf{1}\) for all \(A\), we say that \(F\) is a tautology;
        and we write \(\models F\) to denote it. 
        Viceversa, if \(A(F) = \mathbf{0}\) for all \(A\), we say that \(F\) is a contradiction. 
        In this case we write \(\models \lnot F\) to denote it. 
        If \(A(F) = \mathbf{1}\) for some \(A\), we say that \(F\) is satisfiable.

    As the reader may be aware of, the satisfiability problem, i.e., the problem of 
        enstablishing if a given formula is true, is NP-complete\footnotemark.

    \footnotetext{Throught this notes we assume the reader to have a background in complexity theory.}
\end{document}
