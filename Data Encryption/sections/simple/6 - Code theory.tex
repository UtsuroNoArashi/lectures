\documentclass{subfiles}
\begin{document}
    Code theory, alongside Information theory, is at the core of modern 
        computer science; the former, due to Richard Hamming, deals with 
        how informations are encoded and corrected when errors occur during trasmission,
        the latter, due to Claude E. Shannon, deals with communication channels,
        protocols, and limits of computability.

    In what follows, we assume the reader has a basic knowledge of linear algebra,
        as many reference to it will be made. In addition to that, some definitions
        are needed to properly follow what is here explained. 

    \begin{definition*}
        Let \(A\) be a finite set of symbols; \(A\) is called an alphabet.
    \end{definition*}

    \begin{definition*}
        Let \(A\) be an alphabet, and \(n\) a positive integer. The set 
        \[
            A^{n} = \set{(a_{1}, a_{2}, \ldots, a_{n})}[ a_{1}, a_{2}, \ldots a_{n} \in A]
        \]
        represents the set of words over \(A\). For \(\overline{w} \in A^{n}\) we define 
        the weight \(\norm{\overline{w}}\) of \(\overline{w}\) as the number of non-zero 
        coeffient of \(a_{i}\) in \(\overline{w}\). In addition, \(\norm{\overline{w}} = 0
        \iff \overline{w} = \overline{0}\).
    \end{definition*}

    \begin{definition*}[Hamming distance]
        Consider \(\overline{w_{1}}, \overline{w_{2}} \in A^{n}\). 
        We define their (Hamming) distance as the number of coeffients for which these differ. 
        That is, 
        \[
            d(\overline{w_{1}}, \overline{w_{2}}) = \norm{\overline{w_{1}} - \overline{w_{2}}}.
        \]
        It's easily shown that Hamming distance is a metric.
    \end{definition*}

    \begin{definition*}
        We call \(\mathcal{C} \subseteq A^{n}\) a code. In addition, if 
            \(\mathcal{C}\) is a vector subspace of \(A^{n}\), 
            we say that \(\mathcal{C}\) is a linear code.
    \end{definition*}

    In the remaining part of this section, we often use the notation of \([n, k, d]_{q}\) code,
    in which \(n\) is the length of the strings, \(k\) the dimesion of the code, 
    \(d\) the minimal distance and \(q\) the number of elements in \(A\).

    \subsection{Code representations}
    \subfile{../sub/6.1 - Code representations.tex}
\end{document}
