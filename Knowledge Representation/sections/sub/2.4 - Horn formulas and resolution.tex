\documentclass{subfiles}
\begin{document}
    \begin{definition*}
        Let \(F\) be a formula. We say that \(F\) is a literal if 
            it's either an atomic formula, or the negation of an atomic formula.
    \end{definition*}

    \begin{definition*}
        Let \(F\) be a formula. We say that \(F\) is in \gls{cnf}
            if it can be written as conjunction of disjunctions. That is, if 
            \[
                F = \bigwedge_{i = 1}^{n} 
                    \left\lparen \bigvee_{j = 1}^{m} L_{i, j}\right\rparen
            \]
            where \(L_{i, j}\) is a literal.
    \end{definition*}

    It easy to understand that not all formulas are given in CNF. Thankfully there exists 
        an algorithm that allows one to convert each non-CNF formulas into the 
        equivalent CNF. We illustrate such algorithm in the following.

        Let \(F\) be a non-CNF formula, to get the equivalent CNF:
        \begin{description}
            \item [\textbf{Step 1:}] Replace each subformula of the form 
                \(G \implies H\) and \(G \iff H\) with their expanded counterparts. 
                That is, replace each \(G \implies H\) with \(\lnot G \lor H\) and 
                all \(G \iff H\) with \((\lnot G \lor H) \land (\lnot H \lor G)\).

            \item [\textbf{Step 2:}] Get rid of all double negations and apply 
                De Morgan's rules. That is, replace 
                \[\begin{aligned}
                    & \lnot\lnot G \text{ with } G, \\ 
                    & \lnot(G \land H) \text{ with } \lnot G \lor \lnot H, \text{ and} \\ 
                    & \lnot(G \lor H) \text{ with } \lnot G \land \lnot H.
                \end{aligned}\]

            \item [\textbf{Step 3:}] Distibute over \(\land\). That is, given 
                \(G \land (H \lor K)\) replace it with \((G \land H) \lor (G \land K)\).
        \end{description}

    If a formula \(F\) is in CNF and every disjunction has at most one positive literal,
        that is, all literals but one are negated, we say that \(F\) is a Horn formula.
        Among other things, Horn formulas admits a solution to the satisfiability problem.

    \subsubsection{Resolution}
    \subfile{../subsub/2.4.1 - Resolution}
\end{document}
