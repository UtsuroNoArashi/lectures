\documentclass{subfiles}
\begin{document}
    One of the simplest approaches is to employ the binary representation of the 
    given integers. However, there is a limitation: such a representation is not 
    prefix-free.

    A more formally valid, though still not efficient on its own, method is the 
    so-called \emph{unary encoding}. Specifically, given an integer \(x\), we 
    encode it as a sequence of \(x - 1\) zeros followed by a one. It is evident 
    that this representation is prefix-free; however, it is not practical, as the 
    number of bits required grows linearly with \(x\).

    Clearly, an alternative is required. Although numerous codes exist for 
    encoding integers, the following sections concentrate on the \emph{Elias codes} 
    and the \emph{Fibonacci codes}.
\end{document}
