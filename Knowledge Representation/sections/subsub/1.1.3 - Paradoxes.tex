\documentclass{subfiles}
\begin{document}
    The concept of paradox is well known to everybody, 
        though in the context of logic it's usually missleading;
        by this we refer to the fact that some of those that are generally considered paradoxes, are not.
        For instance, let us consider Epimenides's liar paradox:
        \begin{quote*}
            All Cretans always lie.
        \end{quote*}
        Since Epimenides was Cretan, he himself would be lying.
        Thus, the phrase ``All Cretans always lie'' cannot be true, 
        cause it would contradict itself. Therefore, it's simply false.

    More often then not paradoxes derive from the fact that semantic meaning is attributed 
        to a sentence that refers to itself. This is what we call self-referentiality.
        This problem with self-referentiality was later understood by \emph{Russell}.
\end{document}
