\documentclass{subfiles} 
\begin{document}
    Let's now focus our attention onto the source itself.
    In general, the alphabet these use may not be suitable for transmission,
        for some reason or another. For this particular reason, 
        an \emph{encoder} is used to convert the source alphabet,
        to a new one, which is more suitable for transmission.
        On the recieving side, 
        a \emph{decoder} is used to convert back to the original alphabet.
        This process is called \emph{source encoding/decoding}.

    Formally, given a source \(S\) defined on some alphabet, 
        and \(X\) a new alphabet, called \emph{input alphabet}, 
        we define a function \(C : S \to X\) that maps sequences of symbols of \(S\)
        to sequences of symbols in \(X\). 
        A sequence of symbols in \(X\) is called a \emph{codeword}.

    Before we consider any particular \emph{code},
        let us consider the general case. 
        Let's \(C\) be a generic mapping from a source \(S\) to some new alphabet \(X\).
        Since we have imposed no condition on \(C\), 
        one may define it in such a way that two, or more,
        source symbols share the same codeword.
        It easy to observe that, in doing so,
        the decoded message may not be correct or even unique.

    \begin{example*}
        Let's consider the code shown below.
        \subfile{../../extra/TikZ/Figure * - Example of non non-singular codes}
        How should we decode the text 000111? 
            We have two possibilities: 
                either as \(s_{1}s_{1}s_{3}s_{2}\) or as \(s_{1}s_{1}s_{3}s_{4}\).
        As said before, the decoded message might not be unique. 
        Also, unless context is given, there's no way to know which one is correct.
    \end{example*}

    From the above example, we can conclude that a ``good'' code must encode each 
    source symbols with a unique codeword. 
    We call such codes \emph{non-singular codes}.
  
    \subsubsection{\Gls{ud}}\label{Sec:1.2.1}
    \subfile{../subsub/1.2.1 - UD codes}

    \subsubsection{Sardinas-Patterson algorithm}
    \subfile{../subsub/1.2.2 - Sardina-Patterson}

    \subsubsection{Prefix codes}
    \subfile{../subsub/1.2.3 - Prefix codes}
\end{document}
