\documentclass{subfiles}
\begin{document}
    Assume \(\mathcal{C}\) to be represented by means of parametric equations.
        This means that \(\mathcal{C}\) can be written as linear combination
        of \(k\) linearly indipendent vectors of \(A^{n}\), that is,
        \[
            \mathcal{C} = \set{\overline{c} \in A^{n}}[%
                \overline{c} = \alpha_{1}\overline{w_{1}} + \cdots 
                + \alpha_{k} \overline{w_{k}}]
        \]
        with \(\alpha_{1}, \ldots, a_{k} \in A\) and \(\overline{w_{1}}, \ldots, \overline{w_{k}}\)
        vectors generating \(A^{n}\).
    
    If we denote by \(G\) the matrix of the coefficients of the \(k\) linearly indipendent vectors, 
        any word \(\overline{c} \in \mathcal{C}\) can be written as \(\overline{c} = \overline{\alpha}G\)
        with \(\overline{\alpha} \in A^{k}\).
\end{document}
