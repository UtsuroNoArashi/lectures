\documentclass{subfiles}
\begin{document}
    In 1982, \emph{Storer \emph{and} Szymanski} observed that,
    while compressing with LZ77 two situation may occur: a match is found, or it has not.
    In either case the use of a third component makes no sense.
    In \cite{storer1982} they propose a variant of LZ77, the LZss,
    that instead of the triples, make use of pairs of the type: 
    \(<d, \abs{\alpha}>\) when a match is found, or \(<0, c>\) for when it has not.

    Besides this change, the rest of the algorithm is analogous to LZ77.
    
    \paragraph{Comparison of LZ77-type parsings}
    Let us denote by \(z\) the number of phrases in LZ77 encoding, and by \(z'\) those of LZss.
    Define LZ77-/LZss-type parsings, respectively, as follow.

    \begin{description}
        \item [\textbf{LZss-type parsing}] Let \(S = t_{1}t_{2}\ldots t_{r}\) be a decomposition of 
            \(S\) into non-empty strings \(t_{1},t_{2}, \ldots, t_{r}\). 
            We say that \(t_{1},t_{2}, \ldots, t_{r}\) is an LZss-type parsing 
            if of each \(i \in [1, r]\), the string \(t_{i}\) is either a letter or 
            has an occurence in the string \(s[1,\abs{t_{1},t_{2}, \ldots, t_{r}} - 1]\).

        \item [\textbf{LZ77-type parsing}]  Let \(S = t_{1}t_{2}\ldots t_{r}\) be a decomposition of 
            \(S\) into non-empty strings \(t_{1},t_{2}, \ldots, t_{r}\). 
            We say that \(t_{1},t_{2}, \ldots, t_{r}\) is an LZ77-type parsing
            if of each \(i \in [1, r]\), the string \(t_{i}[1, \abs{t_{i}} - 1]\) 
            has an occurence in the string \(s[1,\abs{t_{1},t_{2}, \ldots, t_{r}} - 2]\).
    \end{description} 

    We get the following results.
    \begin{lemma*}[see \cite{lempel1976}]
        For any string \(S\):
        \begin{enumerate}
            \item \(z' = \abs{LZss(S)}\) is smaller or equal to the size of any LZss-type
                factorizations.
            \item \(z = \abs{LZ77(S)}\) is smaller or equal to the size of any LZ77-type 
                factorizations.
        \end{enumerate}
    \end{lemma*}

    \begin{lemma*}[\cite[Lemma 3]{kosolobov2019}]
        For any string \(S\), it holds 
        \[
            \frac{1}{2}z' < z \le z'
        \]
    \end{lemma*}

    \begin{proof*}
        Let us consider the following: given \(S\) a string, let \(f_{1}f_{2}\ldots f_{z}\)
        be its LZ77 parsing. Consider \(f_{1}t_{2}t_{2}'\ldots t_{z}t_{z}'\) where 
        \(t_{i} = f_{i}[1, \abs{f_{i}} - 1]\) and \(t_{i}' = f_{i}[\abs{f_{i}}]\).
        One can easily observe that the latter is a LZss-type parsing of size at most \(2z - 1\).
        Hence \(z' < 2z\). Moreover, the LZss parsing of \(S\) is a LZ77-type factorizations,
        therefore \(z \le z'\).             
    \end{proof*}
\end{document}
