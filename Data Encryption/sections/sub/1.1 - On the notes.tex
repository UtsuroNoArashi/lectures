\documentclass{subfiles}
\begin{document}
    These notes will focus on both Cryptography and Code theory,
        or at least, some parts of them. 

    More precisely, we begin by introducing some fundamental concepts of abstract algebra 
        and projective geometry, we then discuss what public key encryption is 
        focusing on the RSA and the ElGamal algorithm; 
        we continue with describing some algorithms to factor integers, 
        and conclude with several types of codes.

    The following notation will be used throughout the document.
    \begin{itemize}
        \item \(\mathcal{A}, \mathcal{B} \text{ and } \mathcal{E}\) will indicate, 
            respectively, the sender, the reciever and an unauthorized individual 
            attempting to capt our message.

        \item \(\C*\) will denote codes.

        \item Words in a code \(\C*\) will be represented by \(\mathbf{w}\),
            with the occasional use of subscripts.
    \end{itemize}
    Any other symbol will be explained beforehand.
\end{document}
