\documentclass{subfiles}
\begin{document}
    Before presenting Fibonacci codes, recall that Fibonacci numbers are defined 
    by the recurrence
    \[
        F_n = 
        \begin{cases}
            0, & n = 0; \\
            1, & n = 1; \\ 
            F_{n-1} + F_{n-2}, & n \ge 2
        \end{cases}
    \]

    We also introduce a fundamental result essential for Fibonacci codes.

    \begin{theorem*}[Zeckendorf]
        Any positive integer can be uniquely expressed as the sum of 
        non-consecutive Fibonacci numbers.
    \end{theorem*}

    Using this theorem, we can construct a code that efficiently encodes 
    integers. Let \(n\) be a positive integer. The encoding procedure is as 
    follows:
    \begin{enumerate}
        \item Find the largest Fibonacci number less than or equal to \(n\).

        \item Suppose it is the \(i\)-th Fibonacci number. Subtract it from 
              \(n\) and record the remainder. Set the \((i-1)\)-th bit to 1 
              (the leftmost bit has index 0).

        \item Repeat the above steps, replacing \(n\) with the remainder, 
              until it becomes 0.

        \item Append an additional 1 to the right end of the codeword.
    \end{enumerate}

    \begin{example*}
        For \(n = 73\), it can be expressed as 
        \(F_{10} + F_7 + F_5\). Applying the above procedure, the Fibonacci 
        encoding of \(n\) is 
        \[
            0001010011.
        \]
    \end{example*}

    Decoding is straightforward. Remove the additional 1 at the right end of 
    the codeword. Then, from left to right, replace the \(i\)-th 1 with the 
    \((i+1)\)-th Fibonacci number, and sum all the corresponding Fibonacci 
    numbers.

    \begin{example*}
        To decode \(0001010011\), first remove the extra bit, obtaining 
        \(000101001\). Then, replace the \(i\)-th 1 with the \((i+1)\)-th 
        Fibonacci number, resulting in
        \[
            F_{5} + F_{7} + F_{10} = 73.
        \]
    \end{example*}
    \clearpage
\end{document}
