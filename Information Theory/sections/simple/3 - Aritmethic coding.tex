\documentclass{subfiles}
\begin{document}
    Let us recall that, given \(C\) a code, 
        \(L_{S}(C)\) represents the average code length of the codewords in \(C\).

    From it, we can define two other important quantities: 
        the efficiency and the redundancy of the code.
        Formally, we define the efficiency as 
        \[
            \vartheta(C) = \frac{\Entropy[d]{S}}{L_{S}(C)}
        \]
        which, from Shannon's theorem, is a value in the range \([0, 1]\).
        We define the redundancy of the code as 
        \[
            \rho(S) = 1 - \vartheta(C)\text{.}
        \]
        When considering Huffman, 
            one can prove that its redundancy approaches 1 when the most 
            frequent symbol probability approaches 1.

        The question then is: can we do better than Huffman?
            The answer is yes, but we have to use a completely different approach:
            instead of assigning a codeword to each symbol, 
            we assign one to the whole text.
            The method we are about to present is known as \emph{\gls{ac}}.

        \subsection{The encoding}
        \subfile{../sub/3.1 - AC encoding}

        \subsection{The decoding}
        \subfile{../sub/3.2 - AC decoding}

        \subsection{The adaptive version}
        \subfile{../sub/3.3 - Adaptive AC}
\end{document}
