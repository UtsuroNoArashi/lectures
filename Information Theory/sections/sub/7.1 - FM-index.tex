\documentclass{subfiles}
\begin{document}
    The FM-index has several applications in computational biology and many other scientific fields.
        The reason behind its success is due the fact that it allows to search and index 
        all the occurences of a given pattern \(P\) efficiently. In fact, as we should show, 
        it takes time \bigO{\abs{P}} to count all the occurences of the pattern, 
        time \bigO{\abs{P} \log^{1 + \varepsilon} n}, where \(\varepsilon\) is an arbitrary positive constant chosen 
        during the construction of the index, to locate them and at most 
        \(5n \ \entropy{T}[k] + \bigO{\sfrac{n}{\log^{\varepsilon} n}}\) bits of space.

    The algorithm operates under the assumption that the input text has been compressed through 
        a specific sequence of preprocessing steps. First, the BWT is applied to the original text.
        The resulting sequence is then processed using the MTF transform, followed by RLE. 
        Finally, a variable-length prefix coding scheme is employed to produce the compressed representation.
        We refer to such a compressor as \lstinline{BW_RLX}, keeping the same notation used by Ferragina and Manzini.

    We briefly recall how MTF and RLE work.
    \begin{description}
        \item [\textbf{MTF:}] Consider the array \(MTF[0, \abs{\Sigma} - 1]\) lexicographically sorted. 
            Replace character \(c\) in the text with the number of distinct characters seen since the previous occurence of \(c\).
            Move \(c\) to the front of the \(MTF\) array.
            For instance, let \(MTF[0, 3]\), let \(S = abbcdcdb\) and assume the lexicographic order to be the alphabetical one
            (\(a < b < c < d\)). Applying the MTF to \(S\), we denote it by \(S_{MTF}\), we have \(S_{MTF} = 01023112\).

        \item [\textbf{RLE:}] Consider a string \(S\) with some runs\footnotemark in it.
            Replace each run with its length and the character. For instance,
            let \(S = \$iiiimppssss\), its RLE is \(S_{RLE} = \$4im2p4s\).
            
            Let us consider a variant. We use such variant for the rest of this section. 
            Consider a string with runs of zeros. Any such sequence can be wrote as \(0^{k}, k \in \Z^{+}\).
            Consider the representation of \(k + 1\) in binary. Swap the most and the least significant bits,
            after that drop the new least significant bit. Replace the run with this representation of \(k + 1\).
    \end{description}

    \footnotetext{In this context, by ``run'' we refer to a sequence of the same symbol.}
\end{document}

%% BT_i are the partitions of BWT(T).
%% BT^{MTF}_i are the partitions induced by the one above on MTF(BWT(T))
%% BZ_i partitions of the compressed text. 
