\documentclass{subfiles}
\begin{document}
    Signing a document by means of the RSA algorithm is somewhat straightforawrd.
        Let \(d\) is the hash value of some document \(\mathcal{A}\) produced;
        to sign \(d\) she proceed as follow: she chooses two primes \(p \text{ and } q\),
        computes their product \(n = pq\) and publish \(n\) and a public verification exponent.

    Following a similar reasoning to that of \emph{Section \ref{Sec:3.2.1}}, the congruence 
        \[
            sv \equiv 1 \mod{(p-1)(q - 1)}
        \]
        is solved with respect to \(s\). Here \(s\) is the \emph{signing exponent} and \(v\)
        the \emph{verification exponent}. Lastly, the document is singned as
        \[
            \mathcal{S} \equiv d^{s} \mod{n}.
        \]
        To check the signature, \(\mathcal{V}\) simply computes 
        \[
            \mathcal{S}^{v} \mod{n}
        \]
        and checks whether this is \(d\) or not. The correctness of this is given 
        by Euler's formula.


    Similarly to the RSA encryption scheme, the RSA signature securety follows from 
        the hardness of computing the \(v\)-th root of \(d\), 
        rather then the factorization of \(n\).
\end{document}
