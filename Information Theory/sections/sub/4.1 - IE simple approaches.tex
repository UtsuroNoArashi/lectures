\documentclass{subfiles}
\begin{document}
    One of the simplest approch is to use the binary reprentation of the given integers,
    but there's an issue: such reprentation is not prefix free.

    A more valid, but still not reasonable on its own, is the so called \emph{unary encoding}.
    Briefly, given \(x\) an integer we encode it as a sequence of \(x - 1\) 0s followed by a 1.
    It obvious that such reprentation is indeed prefix free, 
        but it's not reasonable -- the higher the integer, 
        the higher is the number of bits needed for its reprentation. 
    
    Clearly we needs something else. Even though there's plenty of codes to encode integers, 
    the following sections focus on the \emph{Elias codes} and the \emph{Fibonaci codes}. 
\end{document}
