\documentclass{subfiles}
\begin{document}\label{Sec:2.2}
    ElGamal algorithm is somewhat an extention of the Diffie-Hellman protocol;
        in fact, apart from the last additional step, both work the same.
        More precisely, a TTP chooses a 128 bits, sometimes even 256 or 512 bits, 
        prime \(p\) and fixes a \(\gamma \in \gf{p}\). 
        As for Diffie-Hellman, both A and B choose a private key, say \(a\) and \(b\),
        and compute their public key \(\alpha = \gamma^{a} \mod{p}\) and 
        \(\beta = \gamma^{b} \mod{p}\) respectively.

    Once \(k\) has been choosed, assuming B wants to send a text \(t\) to A,
        they compute the cypher \(c = kt \mod{p}\), and sends it to A.
        On the recieving side, A decodes \(t\) by computing \(t = k^{-1}c \mod{p}\),
        where \(k^{-1}\) is the multiplicative inverse of \(k \mod{p}\).

    \begin{example*}
        Let \(p = 47\) and \(\gamma = 2\). We have that 
        \[
            \gf{47} = \set{0, 1, 2, \ldots, 46}.
        \]

        Say A chooses \(a = 17\), thus computing their public key,
            they get \(\alpha = 2^{17}\).
        Say B chooses \(b = 31\), therefore computing the public key,
            they get \(\beta = 2^{31}\).

        It's easy to see that \(k = 12 \mod{47}\). Hence, 
            if \(t = 40\) is the text to encrypt,  
            B cyphers it as \(c = km = 10\) and sends it to A.
            A recieves \(c\) and decodes \(t = 40\) by computing 
            \(k^{-1} = 33\).
    \end{example*}

    \subsubsection{Complexity of ElGamal}
    \subfile{../subsub/2.2.1 - Complexity of ElGamal}
\end{document}
