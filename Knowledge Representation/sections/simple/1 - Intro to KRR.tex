\documentclass{subfiles}

\begin{document}
    \gls{krr} is a branch of Artificial Intelligence focused on understandig how 
        knowledge, whose meaning will be soon clarified, 
        can be represented symbolically and manipulated automatically by reasoning programs.

    Formally, knowledge is defined as the relationship existing between a \emph{knower} and 
        a proposition; that is, the idea expressed by a simple declarative sentence
        (eg. ``Mark will come to the party'').
        To represent and manipulate knowledge, which is our end goal, we have to understand 
        what reasoning is.
    Let \(p_{1}, p_{2}, \ldots, p_{n}\) be a sequence of premises;
        we define \emph{reasoning} as the process which, from the premises \(p_{1}, \ldots, p_{n}\)
        allows one to draw a conclusiion. Later on we discuss the logic formalization of reasoning.

    Throughout this notes we use a ``logic language'', later discussed, known as \gls{fol}.
        Similarly to any other language it has its own:
        \begin{itemize}
            \item \emph{syntax}: the set of symbols and rules of the language;
            \item \emph{semanic}: the meaning of syntactically correct sentences; and
            \item \emph{pragmatics}: how contex and habits contribute to the meaning.
        \end{itemize}

    The next two sections cover all the type of reasoning. 
        Precisely, we first discuss of all those aspects of the human reasoning that lead to an incorrect reasoning,
        we then focus on the three main types of correct reasoning.

    \subsection{Sources of incorrect reasoning}
    \subfile{../sub/1.1 - Sources of incorrect reasoning}

    \subsection{Correct forms of reasoning}
    \subfile{../sub/1.2 - Correct forms of reasoning}
\end{document}
