\documentclass{subfiles}
\begin{document}
    Kolmogorov complexity is defined as the smallest program a Turing machine has to execute 
        to produce a string \(x\). This model is way too powerful to be exploited fully;
        can we therefore use some other model, and if so, which one?
        Answer to this question are the \gls{cfg}, since these allow to capture regularities 
        within a text.

    The use of CFG leads to the following problem, known as \emph{smallest grammar problem}:
        given a string \(x\), which is the smallest grammar that produces exactly \(x\)?
        In this context, the complexity is given by the size of the grammar, which in turns,
        is given by the number of symbols on the right-hand side of all rules; that is,
        given \(G = (\Sigma, V, P, S)\) a CFG, with \(P = \set{A_{1} \to v_{1}, \ldots, A_{k} \to v_{k}}\),
        \(\abs{G} = \sum_{i=1}^{k}{\abs{v_{i}}}\).

    Let us observe that going from the Turing machine model to the CFG one has reduced the complexity
        to mere intractability; in fact, one can prove that SGP is NP-complete.
        Therefore, the best we can hope for is an approximation of the ``grammatical complexity''.

    Let \(A\) be algorithm such that \(A(x)\) is a grammar. We define 
        \[
            \rho_{A}(n) = \max\limits_{x \in \Sigma^{n}} \frac{\abs{A(x)}}{\abs{G_{min}(x)}}
        \]
        where \(G_{min}(x)\) is the smallest grammar that produces \(x\).
        In \cite{charikar2005} , Charikar et al. proved that, unless P = NP, \(\rho_{A}(n) < \sfrac{8569}{8568}\).
        At date, the best algorithm to approximate the SGP are SEQUITUR (\cite{witten1997}) and RE-PAIR (\cite{larsson2000}).
\end{document}
