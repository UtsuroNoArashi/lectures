\documentclass{subfiles}
\begin{document}
    Let \(x\) be a positive integer. Its delta code, denoted \(\delta(x)\), 
        is a binary sequence composed of the following elements:
        \begin{itemize}
            \item the gamma encoding of \(\abs{B(x)}\), and
            \item the binary representation of \(x\) with the most significant 
                  bit excluded.
        \end{itemize}

    Decoding is straightforward: count the number of zeros up to the first 1, 
        say they are \(k\); treat the next \(k + 1\) bits as the integer \(h\);
        interpret the following \(h\) bits as the integer \(x\).

    \begin{example*}
        For \(x = 11\), the delta encoding is
        \[
            \delta(x) = 00100 011,
        \]
        where \(00100\) corresponds to the gamma encoding of \(\abs{B(x)}\) 
        and \(011\) represents the binary encoding of \(x\) with the most 
        significant bit excluded.
    \end{example*}

    It can be shown that 
        the delta code requires at most 
        \[
            1 + \log x + 2 \log \log x
        \] 
        bits. Importantly, delta codes are efficient when the probability 
        distribution is 
        \[
            p(x) = \frac{1}{2 x (\log x)^{2}}.
        \]
\end{document}
