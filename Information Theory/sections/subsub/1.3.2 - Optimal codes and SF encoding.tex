\documentclass{subfiles}
\begin{document}
    The concept of average code length allows us to define an interesting class of  codes:
        the \emph{compact} (or optimal) codes.
        These are UD codes that have the lowest ACL.

    A first attempt to achieve such codes was proposed by Shannon and \emph{Robert Mario Fano},
        who developed the so-called Shannon-Fano encoding.

    \paragraph{Shannon-Fano encoding}
    We will consider the binary case. 
        The encoding itself is very simple: we order the symbols in decreasing order,
        divide the symbols into two sets such that the sum of probabilities in each set is almost equal, 
        encode the symbols in the first set with 0 and the other with 1,
        and finally repeat the procedure recursively for each set.

    \begin{example*}
        Consider the alphabet \(\Sigma = \set{a, b, c, d}\) and let \(p_{a} = \sfrac{1}{2},
            p_{b} = \sfrac{1}{8}, p_{c} = \sfrac{1}{4}\) and \(p_{d} = \sfrac{1}{8}\)
            Applying the steps above, we get the codes shown in the table below.
            Here, each color is used to show a different iteraction.
            \subfile{../../extra/TikZ/Tab * - Example of Shannon Fano encoding}
    \end{example*}

    One can prove that Shannon-Fano encoding is not optimal. We present it just for hystorical reasons.
    \cleardoublepage
\end{document}
