\documentclass{subfiles}
\begin{document}
    Sequence alignment is the process trough which we search for patterns 
        within the given sequences. The process can be either applied locally,
        we discuss in details \cite{smith1981}, or globally, we present a variant of \cite{needleman1970}.
        Before doing so, we consider the na\"ive approach; that is, 
        we just take in consideration mismatches and matches. 

    Considering two sequences, essentially, we consider all the possible alignments 
        and choose the one with the highest number of matches.
        \begin{example*}
            Let \(s_{1} = ACCAC\) and \(s_{2} = ACGGC\). Consider all possible alignments:
            {\setlength{\jot}{0pt}
            \begin{minipage}{0.45\textwidth} 
            \[\begin{aligned}
                    & ACCAC \\
                    & ACGGC \\ 
                            \\ 
                    & ACCAC \\ 
                    A &CGGC \\
                            \\ 
                    & ACCAC \\ 
                    AC &GGC \\ 
                            \\ 
                    & ACCAC \\ 
                    ACG& CA \\ 
                            \\ 
                    & ACCAC \\
                    ACGG &C \\
            \end{aligned}\]
            \end{minipage}
            \begin{minipage}{0.45\textwidth}
                \[\begin{aligned}
                    A& CCAC \\
                    & ACGGC \\ 
                            \\
                    AC &CAC \\ 
                    & ACGGC \\
                            \\
                    ACC &GC \\
                    & ACGGC \\ 
                            \\ 
                    ACCG &C \\ 
                    & ACGGC \\ 
                \end{aligned}\]
            \end{minipage}}
            
            It's easy to observe that the best one is the top-left one.
        \end{example*}
    This approach has a severe issue: it does not take in account gaps,
        which, as we have said, are extremely frequent in nature. Therefore,
        though easy to inplement, we should consider something different.
    
    The necessity to consider gaps lead to a question: how do we define ``the best alignment``
        when gaps are involved? To solve the problem we define the so called \emph{similarity measure};
        i.e., a measure that takes in account both matches/mismatches and gaps.
        The one we use is the following
        \[
            S_{AB} = \sum_{i = 1}^{L}{s(a_{i}, b_{i})} - \sum_{k = 1}^{NG}{\delta +\gamma[l(k) - 1]}
        \]
        where \(\delta\) is said to be the \emph{opening penalty}, \(\gamma \le \delta\) is 
        called \emph{extention penalty} and \(l(k)\) is the length of the \(k\)-th gap.
    
    Let us note that the concept of match/mismatch holds just for DNA and RNA sequences;
        in the context of amino acids, and therefore proteins, we a different criterion:
        the \emph{amino acid interconvertibility} criterion. In this case we make use of 
        two special matrices (more correctly, substitution matrices) which tells us the value to assign
        to any given pair of amino acid. The ones we consider are PAM-1 and BLOSOM-62.
\end{document}
