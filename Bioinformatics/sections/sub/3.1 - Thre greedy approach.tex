\documentclass{subfiles}
\begin{document}
    As we saw, the use of dynamic programming is inconcivable for multiple alignment.
        In this section we focus on heuristic algorithms that compute the best alignment 
        in by means of a greedy approach. Roughly, algorithms procced as follow:
        let \(w_{1}, w_{2}, w_{3}, \ldots, w_{n}\) be strings to align.
        By some heuristical method choose \(w_{i}, w_{j}, i \ne j\) such that 
        these are the most similar; combine them to reduce the alignment from \(n\)
        sequences to \(n - 1\). For instance, let 
        \[\begin{aligned}
            w_{1} & = ABBBCALLTA \\ 
            w_{2} & = GCGAVBGHAK \\ 
            w_{3} & = LBBBJALLTA \\
        \end{aligned}\]
        be the sequences to align, then we can combine \(w_{1}, w_{3}\) in the string 
        \(w_{1,3} = a/lBBBc\/jALLTA\), therefore reducing the alignment to the sequences 
        \(w_{2}, w_{1,3}\).

    algorithms that use this approach are known as progressive alignment algorithms.

    \subsubsection{ClustalW}
    \subfile{../subsub/3.1.1 - ClustalW}

    \subsubsection{Scoring functions for multiple alignment}
    \subfile{../subsub/3.1.2 - Scoring function for multiple alignment}
\end{document}
