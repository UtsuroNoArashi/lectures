\documentclass{subfiles}
\begin{document}
    The concept of \emph{fallacy} is straightforward.
        It is a argument that, despite sounding convincing, is invalid.

    \begin{example*}
        Consider the sentece ``A pencil isn't an animal. All animal feeds. 
            Therefore a pencil doesn't feed''.
            Though both premisies are true, and therefore the conclusion,
            the overall sentece is non-sensical.
    \end{example*}

    We can distinguish several types of fallacies, some of which are listed below.
    \begin{itemize}
        \item \underline{Fallacy of composition} (\foreignlanguage{latin}{compositio}): 
            attributes to an entity as a whole some property valid for its parts.
            \begin{quote*}             
                The product is on sale at a price of 10,000 euros, payable in small
                installments of only 100 euros each. Therefore the total price is
                moderate.
            \end{quote*}

        \item \underline{Fallacy of division} (\foreignlanguage{latin}{divisio}):
            attributes to the individual parts a property that is valid for the whole entity 
            \begin{quote*}
                A truck is heavy. Therefore each of its component is heavy.
            \end{quote*}

        \item \underline{Fallacy of relevance}:
            derives a conclusion from premisies that are irrelevant.

            It has different forms. We list them below.
            \begin{itemize}
                \item \underline{\foreignlanguage{latin}{Argumentum ad hominem}}:
                    discredits a thesis in order to deduce its falsity.

                \item \underline{\foreignlanguage{latin}{Tu quoque}}:
                    the one proposing a thesis does the opposite of what he wants to prove.

                \item \underline{\foreignlanguage{latin}{Ad vericundum}}:
                    the thesis is supposed to be true just because credit is given to those proposing it.

                \item \underline{\foreignlanguage{latin}{Ad populum}}:
                    accepts a thesis simply because it's accepts by the majority.
                
                \item \underline{Straw Man}:
                    creates a distorted version of a thesis, to refute it.

                \item \underline{\foreignlanguage{latin}{Ad ignorantiam}}:
                    supports a thesis by refuting the arguments in favor of the opposite thesis.

                \item \underline{\foreignlanguage{latin}{Petitio principi}}: 
                    the thesis itself is hidden within the premisies.

                \item \underline{False dichotomy}: present two solution to the problem, as if these are the only ones.

                \item \underline{\foreignlanguage{latin}{Non causa pro causa}}: 
                    fix two peculiar characteristics and derive that the first causes the second one.
            \end{itemize}
    \end{itemize}
\end{document}
