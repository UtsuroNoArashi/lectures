\documentclass{subfiles}
\begin{document}
    Let \(A\) and \(B\) be two entities who want to communicate. 
        Both require that no one but them can understand what the communication is about.
        One of the many solutions is the Diffie-Hellman algorithm, 
        which is based on the hard problem of computing the discrete logarithm.
    
    The algorithm proceeds as follow:
        \begin{enumerate}
            \item Assume a \gls{ttp} exists; that is, assume that there exists 
                a third ``person'', besides A and B, that acts as mediator between A and B.

            \item This TTP chooses a \(p \in \Z \text{ such that } p\) is prime. 
                Once \(p\) is fixed, consider \(\gf{p}\) and fix a \(\gamma \in \gf{p}\).

            \item Tell A and B to choose a value in \(\set{0, \ldots, p - 1}\), 
                call it \(a\) and \(b\) respectively; this will be their private key.

            \item Let A and B compute their public key as \(\alpha = \gamma^{a},
                \beta = \gamma^{b}\) respectively.

            \item A and B agree on a \(k \in \gf{p}\) such that:
                \[\begin{aligned}
                    \text{A obtains } k \text{ as: } & \beta^{a} \mod{p} = \gamma^{b^{a}} \mod{p} = \gamma^{ab} \mod{p}, \\ 
                    \text{B obtains } k \text{ as: } & \alpha^{b} \mod{p} = \gamma^{a^{b}} \mod{p} = \gamma^{ab} \mod{p}.
                \end{aligned}\]
        \end{enumerate}
        In the steps above \(p\), and therefore \(\gf{p}\), \(\gamma, \alpha\) and 
            \(\beta\) are public.

    Consider a malitious individual, call him E, that wants to intrude in the communication;
        what can he do? Not much actually. In fact, he can only try to compute \(a = \log_{\gamma} \alpha\)
        (or \(b = \log_{\gamma} \beta\) equivalentely). But, recall that \(\alpha, \beta \in \gf{p}\),
        which makes hard to compute the actual value of either \(a\) or \(b\).

    \subsubsection{Complexity of Diffie-Hellman}
    \subfile{../subsub/2.1.1 - Complexity of DH}
\end{document}
