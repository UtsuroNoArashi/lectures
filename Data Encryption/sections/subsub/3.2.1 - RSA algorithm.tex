\documentclass{subfiles}
\begin{document}\label{Sec:3.2.1}
    Let \(m\) be the message to transmit; to cypher it by the RSA protocol do as follow:
        begin by choosing two odd distinc prime numbers \(p, q\) randomly,
        such that their bitlength is approximately the same. 
        Compute \(n = pq\) and \(\lambda = (p - 1)(q - 1)\).
        Let \(\mathcal{B}\) choose an integer \(e\) such that \(\gcd{e, \lambda} = 1\);
        the pair \((n, e)\) will be \(\mathcal{B}\)'s \emph{public key}.
        Let \(d\) be the integer less then \(\lambda, \text{ such that } ed \equiv 1 \mod{\lambda}\);
        this will be \(\mathcal{B}\)'s \emph{private key}. 

    To encode the message \(\mathcal{A}\) computes 
        \[
            c \equiv m^{e} \mod{n}
        \]
        and sends \(c\) to \(\mathcal{B}\). Upon reciving the chypertext, \(\mathcal{B}\)
        can easily decode \(m\) as
        \begin{equation}\label{Eq:1}
            m \equiv c^{d} \mod{n}.
        \end{equation}

    To see why \emph{Equation \eqref{Eq:1}} holds observe the following: since 
        \(ed \equiv 1 \mod{\lambda}\), there exists \(k\) such that \(ed = 1 + k\lambda\).

    Now, we have that:
        \begin{itemize}
            \item If \(\gcd{m, p} = 1\) by Fermat's little theorem 
                \[
                    m^{p - 1} \equiv 1 \mod{p}.
                \]
                Hence, raising both sides by \(k(q - 1)\) and multipling by \(m\) yields
                \[
                    m^{1 + k\lambda} \equiv m \mod{p} \implies m^{ed} \equiv m \mod{p}.
                \]

            \item If \(\gcd{m, p} = p\) the above still holds, as both sides will be congruent to 0.
        \end{itemize}

    Applying the same reasoning for \(q\), since \(p \text{ and } q\) are distinc,
        we conclude that 
        \[
            c^{d} \equiv m \mod{n}.
        \]

    Let us briefly overview why \(\mathcal{E}\) cannot, at least no so easily,
        dechypher the message sent by \(\mathcal{A}\). By the protocol both \(n,
        e\) are know to everyone, but the values of \(\lambda \text{ and } d\) 
        are not as these depend on \(p \text{ and } q\). Hence, the only 
        choice left to \(\mathcal{E}\) is to factor \(n\); 
        which we know is hard to do.

    \begin{example*}
        Let \(p = 37, q = 39\). We have \(n = 1443\) and \(\lambda = 1368\). 
            Say \(\mathcal{B}\) chooses \(e = 263\) for is public key.
            At this point the pair \((n, e) = (1443, 263)\) is know to everybody.
            \(\mathcal{A}\) converts is message into an integer \(0 < m < n\),
            say \(m = 1029\) and computes 
            \[
                c \equiv m^{e} \mod{n} \implies 1320 \equiv 1029^{263} \mod{1443}
            \]
            and sends \(c = 1320\) to \(\mathcal{B}\). In the meanwhile,
            \(\mathcal{B}\) computes \(d\) such that \(eq \equiv 1 \mod{\lambda}\),
            in our case \(d = 671\). Hence, upon reciving \(c, \mathcal{B}\) 
            dechyphers it as 
            \[
                c^{d} \mod{n} = 1320^{671} \mod{1368} = 1029.
            \]
    \end{example*}
\end{document}
