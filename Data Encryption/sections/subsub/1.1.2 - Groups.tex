\documentclass{subfiles}
\begin{document}
    \begin{definition*}
        A \emph{group} is a pair \((G, \cdot)\) where \(G\) is a set and 
            \[
                \cdot : G \times G \to G
            \]
            is a binary operation satisfying:
        \begin{enumerate}
            \item (Associativity) For all \(a, b, c \in G\) we have \((a \cdot b) 
                \cdot c = a \cdot (b \cdot c)\).

            \item (Identity) There exists an element \(e \in G\) such that for all 
                \(a \in G\) we have \(e \cdot a = a \cdot e = a\).

            \item (Inverses) For every \(a \in G\) there exists \(a^{-1} \in G\) with
                \(a \cdot a^{-1} = a^{-1} \cdot a = e\).
        \end{enumerate}

        If, in addition, \(a \cdot b = b \cdot a\) for all \(a, b \in G\), 
            the group is called \emph{abelian} (or commutative).
    \end{definition*}

    \begin{definition*}
        Let \(H \subseteq G\). We say that \(H\) is a subgroup of \(G\), if and only if 
            \(H\) is non-empty and is a group under the operation inherited from \(G\).
    \end{definition*}

    \begin{definition*}
        A group \((G, \cdot)\) is \emph{cyclic} if there exists an element 
        \(g\in G\) such that 
            \[
                G = \set{g^n}[n \in \Z].
            \]
        Such an element \(g\) is called a \emph{generator} of \(G\). 
    \end{definition*}

    \begin{definition*}
        Let \((G, \cdot)\) be a cyclic group, and let \(g \in G\). 
            We call order of \(g\), and denote it as \(\ord{g}\) the smallest integer \(k \in \Z\) such that 
            \(g^{k} = e\).
    \end{definition*}

    Groups are at the core of abstract algebra as they provide the ground of more
        complex structures. We now recall on of the most fundamental theorems in 
        group theory: Lagrange's theorem.

    \begin{theorem*}[Lagrange]
        Let \((G, \cdot)\) be a finite cyclic group, let \(g \in G\).
            Then, 
            \[
                \divides{\ord{g}}{\ord{G}}.
            \]
    \end{theorem*}
\end{document}
