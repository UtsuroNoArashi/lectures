\documentclass{subfiles}
\begin{document}
    One might think that non-singular codes are sufficient, 
        but in most they don't.
        In fact, it can happen that a codeword is a prefix of another,
        making decoding ambiguous or tedious.
    \begin{example*}
        Consider the following code: \(C = \set{0, 1, 01, 11}\). 
            Assume the alphabet of the previous example.
            How should we decode the string 000111? 
            Again, multiple decodings are possible:
            for example \(s_{1}s_{1}s_{3}s_{2}s_{2}\) or \(s_{1}s_{1}s_{3}s_{4}\).
            The correct decoding cannot be determined unless context is given.
    \end{example*}

    We say that a code is UD if and only if each sequence of codewords
        corresponds to at most one sequence of source symbols.
        From this definition, two questions follow:
        \begin{itemize}
            \item How do we construct such codes?
            \item How can we check whether a given code is UD?
        \end{itemize}

    The next section focuses on the second question, 
        while \emph{Sections \ref{Sec:2}-\ref{Sec:4}} focus on the construction of UD codes.
\end{document}
