\documentclass{subfiles}
\begin{document}
    As previously noted, 
        the main limitation of the classic Huffman approach 
        is the necessity to scan the text twice, which slows down encoding.
    Additionally, proper operation requires knowledge of the symbol probabilities in advance,
        which is not always feasible.
    To address these issues, a new method has been developed, 
        known as \emph{Adaptive Huffman}.

    The key feature of this approach is that the tree is built and updated dynamically.
    The procedure can be summarized as follows:
    \begin{itemize}
        \item Start with an initially empty tree,
            or one containing only a special \textbf{Not Yet Transferred} (\textbf{NYT}) node.

        \item For each symbol in the text: 
            \begin{itemize}
                \item If the symbol has already been encoded, encode it using the existing code.

                \item If it is a new symbol:
                    encode the \textbf{NYT} node first, 
                    then encode the symbol itself and add it to the tree as a leaf.

                \item After encoding a symbol, update the tree:
                    increment the frequency of the symbol and its ancestors,
                    and reorganize the tree to maintain the Huffman properties.
            \end{itemize}
    \end{itemize}

    It can be shown that the adaptive approach exhibits better locality than the classic one.
\end{document}
