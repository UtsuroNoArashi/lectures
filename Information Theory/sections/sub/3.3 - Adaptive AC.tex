\documentclass{subfiles}
\begin{document}
    It's easy to observe from \emph{Figure \ref{Fig:4}}, 
    that the necessity to know the probabilities a priori,
    slows down the encoding. 
    The question then is: can we improve the AC encoding? 
    The answer is yes, by assuming an initial equality in the probabilities.
    Briefly, before any symbol is read, 
    we assume that each of the alphabet symbol has the same probability to appear within the text.
    Then, at each step, we simply apply the same procedure in \emph{Figure \ref{Fig:4}}
    and then update the probabilities.

    What about the decoding? We can apply the same logic, with a little differce:
    instead of updating the probabilities after a symbols is read, we update them,
    each time a symbol is decoded by applying the procedure in \emph{Figure \ref{Fig:5}}.

    \begin{example*}
        Let \(\omega = abcb\) be the text to encode. 
        Before any character is read, we have the following probabilities: 
        \(p_{a} = p_{b} = p_{c} = \tfrac{1}{3}\).
        Once the first character (\(a\)) is read we select as current range 
        the \([0, \tfrac{1}{3}]\) then we update the probabilities as 
        \(p_{a} = \tfrac{1}{2}, p_{b} = p_{c} = \tfrac{1}{4}\).
        We repeat the same process until the end of the text,
            and encode \(\omega\) as \([\tfrac{43}{180}, \tfrac{11}{45}]\).
        \subfile{../../extra/TikZ/Figure 6 - Adaptive AC encoding}

        Let's now decode \(0.2416\), with \(n = 4\).
        Again we assume equality in the probabilities, 
            and thus, we decode (\(a\)) and update the probabilities.
            Repeating the process, we decode the text \(abcb\).
        \subfile{../../extra/TikZ/Figure 7 - Adaptive AC decoding}
    \end{example*}
\end{document}
