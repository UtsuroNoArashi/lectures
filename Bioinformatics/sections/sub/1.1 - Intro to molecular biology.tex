\documentclass{subfiles}
\begin{document}
    Human bodies, as well as any onther living thing, is constituted of several proteins
        which differ for their purpose. In this sense, we distinguish:
        \begin{itemize}
            \item \textbf{Structural proteins:} serves a building block for cells. 
                An example of these is collagen.

            \item \textbf{Enzymes:} act as ``cathalysts'' for some chemical reaction.
                Some reactions are so slow that, in the case enzymes did not exist, 
                life itself would not exist.

            \item \textbf{Transport proteins:} carry vital substunces through the body,
                i.e., hemoglobin.

            \item \textbf{Antibodies.}
        \end{itemize}

    Considering proteins in general, meaning without caring what's their task, 
        these end up being a chain of smaller molecules: the \emph{amino acids}.
        With reference the figure below,
        \subfile{../../extras/TikZ/Figure . - General structure of an amino acid}
        all amino acids share a common part 
        (the one boxed in \Csquare{rpIris}) and differ for the corresponding side chain
        (a.k.a the R-group, the one boxed in \Csquare{rpPine}).
        Though one could argue the existence of exceptions, we condider the classic 20 side chains,
        and consequentially the 20 classic amino acids (See Appendix A for the whole list).

    In truth proteins are not chains of amino acids, rather their residue: 
        within a proteins, amino acids are bound to each other by petptide bonds, 
        in which the COOH of an amino acid \(A_{i}\) binds with the \(H_{2}N\) of the amino acid \(A_{i + 1}\).
        This bond leads to the release of a water molecule, thus neither of the amino acids has its original 
        molecular structure.

    But how do we get proteins? To answer this question we need to understand what DNA and RNA are.
    
    \subsubsection{From nucleic acids to proteins}
    \subfile{../subsub/1.1.1 - From nucleic acids to proteins}

    \subsubsection{Synthesis}
    \subfile{../subsub/1.1.2 - Synthesis}
\end{document}
