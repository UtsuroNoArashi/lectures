\documentclass{subfiles}
\begin{document}
    Let's look back at the example of \emph{Section \ref{Sec:1.2.1}}. 
        Given the codewords \(\set{0, 1, 01, 11}\) for the source symbols \(s_{1}, s_{2}, s_{3}, s_{4}\) respectively,
        how should we decode the string \(000111\)?
        As noted before, unless some context is given, we cannot be certain.

    Then, what is the issue? 
        Essentially, even though each source symbol is assigned a distinct codeword,
        these codewords are not prefix-free, meaning that some are prefixes of others
        (e.g., the codeword for \(s_{1}\) is a prefix of the codeword for \(s_{3}\)).
        From this observation, a natural solution is to define codes that are prefix-free.
        One can also prove that prefix-free (or simply prefix) codes are also instantaneous;
        that is, each codeword can be immidiately decoded without ambiguity.

    \begin{theorem*}
        Let \(C\) be a prefix-free code; then \(C\) is UD.
    \end{theorem*}

    \begin{proof*}
        It follows directly from \emph{Theorem \ref{Thm:1}}.
    \end{proof*}
\end{document}
