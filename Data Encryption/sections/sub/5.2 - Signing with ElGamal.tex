\documentclass{subfiles}
\begin{document}
    As for the case of RSA, assume that \(A\) is an entity that wants to sign 
        the messages these produce. Again, assume that it has already choosed its 
        private key \(a\). Additionally assume that \(A\), or a TTP, chooses 
        a hash function \(h : M \to \Z_{p - 1}\), where \(M\) is the set of all messages.

    To sign a message \(m \in M, A\) computes \(\overline{m} = h(m) \in \Z_{p - 1}\).
        It then chooses a value \(k\) such that there exist \(k^{-1} \in \Z_{p - 1}\).
        Lastly, computes 
        \[
            s = k^{-1}(\overline{m} - ar) \mod{p - 1},
        \]
        where \(r = \gamma^{k} \mod{p}\). The pair \((r, s)\) is the signature of \(m\).

    Say now \(A\) is asked to prove the ownership of \(m\), what it has to do is compute 
        \[
            v = \gamma^{\overline{m}} - r^{s}\alpha^{r} \mod{p}
        \]
        with \(\alpha\) being A's public key. Note that if \(A\) is the owner of 
        the message, the above will give as a result zero. Following the same 
        reasoning, if \(A\) doesn't own the message the decryption wont succed.
\end{document}
