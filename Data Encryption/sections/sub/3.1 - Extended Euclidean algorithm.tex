\documentclass{subfiles}
\begin{document}\label{Sec:3.1}
    Recall that the classic Euclidean algorithm for some \(a, b \in \Z\), at any step \(i\),
        computes \(q_{i}, r_{i}\) as the integer solution to the equation
        \[
            a = q_{i}b + r_{i} \quad 0 \le r_{i} < b,
        \]
        where \(a = r_{i - 1}, i > 0\); that is, it defines to sequences \(r = \set{r_{i}}_{i \in I}\)
        and \(q = \set{q_{i}}_{i \in I}\).

    The extedend Euclidean algorithm works in a similar manner; in addition to the sequences 
        \(r = \set{r_{i}}_{i \in I}, q = \set{q_{i}}_{i \in I}\), it defines the sequences 
        \(s = \set{s_{0}}_{i \in I}\) and \(t = \set{t_{i}}_{i \in I}\) with the initial condition 
        \(s_{0} = t_{0} = 1, s_{1} = t_{1} = 1\) and \(r_{0} = \max (a, b), r_{1} = \min (a, b)\).
        Then, for \(i = 2, \ldots, n\), compute 
        \[\begin{cases}
            q_{i} = \floor*{\frac{r_{i - 2}}{r_{i - 1}}}, \\ 
            r_{i} = r_{i - 2} q_{i - 1} + r_{i - 1}, \\ 
            s_{i} = s_{i - 2} q_{i - 1} + s_{i - 1}, \\ 
            t_{i} = t_{i - 2} q_{i - 1} + t_{i - 1}. \\ 
        \end{cases}\]
    As for the classic Euclidean algorithm, the computation stops when \(r_{i} = 0\) and \(r_{i - 1} = \gcd{a, b}\).
\end{document}
