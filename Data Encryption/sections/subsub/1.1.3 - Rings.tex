\documentclass{subfiles}
\begin{document}
    \begin{definition*}[Ring]
        A (associative) \emph{ring} is a triple \((R, +, \cdot)\) where \((R, +)\) is an abelian 
            group with identity \(0\), and \(\cdot\) is a binary operation on \(R\) satisfying:
        \begin{enumerate}
            \item (Associativity of multiplication) \(a(bc) = (ab)c\) for all \(a, b, c \in R\).
            \item (Distributivity) \(a (b + c) = ab + ac\) and \((a + b) c = ac + bc\) for all \(a, b, c \in R\).
        \end{enumerate}
        A ring is called \emph{commutative} if \(ab = ba\) for all \(a, b \in R\). 
            A ring has \emph{unity} (or \(1\)) if there exists \(1 \in R\) with 
            \(1 \cdot a = a \cdot 1 = a\) for all \(a \in R\).
        \end{definition*}

    \begin{definition*}[Field]
        A \emph{field} is a commutative ring \((F, +, \cdot)\) with unity \(1 \neq 0\) 
            such that every nonzero element has a multiplicative inverse. Equivalently,
        \[
            (F \setminus \set{0}, \cdot)
        \]
        is an abelian group.
    \end{definition*}
    We denote fields by \(\F\).

    \begin{definition*}[Finite field / Galois field]
        A \emph{finite field}, or \emph{Galois field}, is a field with finitely many elements.
    \end{definition*}
    We will consider multiplicative Galois fields of order a prime, which we denote as \(\gf{p}\).

    Consider the field \(\gf*{p}[x]\), that is, 
        the fields of polynomials with coefficients in \(\gf{p}\).
        Take \(g \in \gf*{p}[x]\), we say that \(g\) is primitive (or irreducible)
        if it cannot be factored as product of other polynomials in \(\gf{p}\). 

    In our discussion the use of irreducible polynomials is due to some codes explained in 
        \emph{Section \ref{Sec:6}}; what matter for the moment is that these allow to 
        extend the field we are considering, to one with more elements. 
        Precisely, let \(g \in \gf*{p}[x]\) be a primitive polynomial.
        For the sake of our discussion assume \(\partial g\) to be at most 3,
        then since it's irreducible, \(g\) also admits no roots. 
        Impose \(i\) to a root for \(g\). 

    In doing so, we've defined a new field \(\gf*{q}[x]\), whose elements are the set 
    \[
        \set{a + ib}[a, b \in \gf*{p}, i \text{ is a root for } g].
    \]
    We call such process, symbolic extension of \(\gf*{p}[x]\).
\end{document}
