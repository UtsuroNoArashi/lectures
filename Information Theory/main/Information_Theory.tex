\documentclass[moon, draft]{lectures2}

\title{Notes on Information Theory}
\author{Riccardo Lo Iacono}

\addbibresource{../extra/Bib/other.bib}
\addbibresource{../extra/Bib/papers.bib}
\addbibresource{../extra/Bib/reports.bib}

\makeglossaries
\newacronym{ud}{UD}{uniquely decodable codes}
\newacronym{acl}{ACL}{average code length}
\newacronym{ac}{AC}{aritmethic coding}
\newacronym{bwt}{BWT}{Burrows-Wheeler Transform}
\newacronym{sa}{SA}{suffix array}
\newacronym{lz}{LZ}{Lempel-Ziv}

\begin{document}
    \maketitle

    \section{Basics of information theory} 
    \subfile{../sections/simple/1 - Basics of IT}

    \section{Huffman encoding}
    \subfile{../sections/simple/2 - Huffman encoding}

    \section{Aritmethic coding}
    \subfile{../sections/simple/3 - Aritmethic coding}

    \section{Integers encoding}
    \subfile{../sections/simple/4 - Integer encoding}

    \section{Compression optimization}
    \subfile{../sections/simple/5 - Compression optimization}

    \section{Dictionary based compressors}
    \subfile{../sections/simple/6 - Dictionary based compressors}

    \section{Pattern matching on compressed indexes}
    \subfile{../sections/simple/7 - PM and CI}

    \printbibliography
    \printglossaries
\end{document}
