\documentclass{subfiles}
\begin{document}
    Decoding a text compressed via AC, is as straightforward as its encoding.
        In fact, the decoding proceeds similarly to the encoding.
        The procedure, shown in \emph{Figure \ref{Fig:5}}, is as follows:
        \begin{enumerate}
            \item Initialize the range \([0, 1]\).

            \item Divide the subrange according to the symbol probabilities,
                as done during encoding.

            \item At each step, select as the current interval the one in which the 
                encoded text falls.
        \end{enumerate}
        \subfile{../../extra/TikZ/Figure 5 - AC decoding snippet}

        \begin{example*}
            Let \(x = 0.3047\). Let the alphabet and the probabilities be those of the previous example.
            \subfile{../../extra/TikZ/Figure 6 - Example of AC decoding}

            \noindent Applying the decoding steps to \(x\) yields what shown in \emph{Figure \ref{Fig:6}}.
        \end{example*}
\end{document}
