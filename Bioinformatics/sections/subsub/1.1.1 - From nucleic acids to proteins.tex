\documentclass{subfiles}
\begin{document}
    In nature there exist two types of nucleic acids: DNA and RNA.
        We present each one briefly. 

    Starting with DNA, as for proteins, this is a chain of simpler molecules:
        the nucleotides. Structurally, it presents as a double strand chain in an helix.
        Chemically speaking (see the figure below), DNA it's a repetion of very similar units.
        \subfile{../../extras/TikZ/Figure . - Structure of DNA units.tex}

    \noindent We distinguish four distinct bases: \gls{a}, \gls{g}, \gls{c} and \gls{t}.
        In DNA adenine bonds with thymine while guanine bonds with cytosine.
        
    Each DNA strand, though bound in pair, follows an orientation: 
        either from 5' to 3' or viceversa. In the following we use the former.
        This mechanism is what allows DNA replication. Essentially, 
        since hydrogen bonds are not covalent (strong) they can be easily broken;
        in addition, since each nucleotide bonds only with a different one,
        from a single DNA strand we can get a copy of it.
        More precisely, given a DNA and a \emph{primer}, that is a segment of DNA that triggers the replication,
        and the four nucleotides, we can synthesize a new complementary strand.

    \clearpage
    RNA is very similar to DNA (see figure below), 
    \subfile{../../extras/TikZ/Figure . - Structure of RNA units}
    and differs from it for the following reasons:
    \begin{itemize}
        \item Structure: RNA is single-stranded;
        \item Thymine is replaced with \gls{u};
        \item We have different types of RNA.
    \end{itemize}

    RNA shows its importance in DNA duplication, as explained in the next section.
\end{document}
