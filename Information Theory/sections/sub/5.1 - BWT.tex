\documentclass{subfiles}
\begin{document}
    The \gls{bwt}, introduced by \emph{M. Burrows \emph{and} D. Wheeler} in \cite{burrows1994},
    is a simple, yet efficient tool to enhance the compressibility of a given string.

    Let \(\omega\) be a string we want to compress, thus preprocess using BWT.
    First step of the transform, is to compute all the cyclic rotations of \(\omega\).
    Then, we sort them lexicographically\footnotemark[3] and lastly, 
    we return the last column. 
    \note{By last column we mean the concatenation of the last symbol of each rotation.} 
    and the index (starting from 0) of the inital string.

    \begin{example*}
        Let \(\omega = aleph\). Its cyclic rotations, already sorted, are
        \[\begin{aligned}
            & aleph \\ 
            & ephal \\
            & halep \\
            & lepha \\
            & phale \\
        \end{aligned}\]
        Thus, we return the pair \((hlpae, 0)\).
    \end{example*}

    Let us point out some interesting properties of the BWT.
        \begin{enumerate}
            \item \(\forall i \ne I \text{, where } I\) is the index of the inital string,
                we have that \(F[i]\) follows \(L[i]\) in \(\omega\).
            \item \(F[I]\) is the first symbol of \(\omega\).
            \item For any character \(x\), the i-th occurence of \(x \text{ in } F\),
                corresponds to the i-th occurence of \(x \text{ in } L\).
        \end{enumerate}

    \footnotetext[3]{For the sake of our disscussion, 
            assume that the lexicographic order is the usual alphabetical one.
            This is, in general, not true.}

    \subsubsection{Inverse of the BWT}
    \subfile{../subsub/5.1.1 - Inverse BWT}
\end{document}

