\documentclass{subfiles}
\begin{document}
    \begin{definition*}
        Let \(a, b \in \Z\) and let \(b \ne 0\).
            We say that \(b\) divides \(a\), and write \(\divides{b}{a}\), 
            if and only if 
            \[
                \exists q \in \Z \text{ s.t. } a = q \cdot b.
            \]
            Viceversa, we say that \(b\) does not divide \(a\), 
            and we write \(\ndivides{b}{a}\).
    \end{definition*}

    \begin{definition*}[GCD]
        Let \(a, b \in \Z\), we say that the integer \(d \in \Z, d > 0\) is the 
            greatest common divisor of \(a\) and \(b\) if the following holds:
            \begin{enumerate}
                \item \(\divides{d}{a}\) and \(\divides{d}{b}\), and 
                \item if \(\exists d' : \divides{d'}{a}\) and \(\divides{d'}{b}\) then \(\divides{d'}{d}\).
            \end{enumerate}
    \end{definition*}

    \begin{definition*}[lcm]
        Let \(a, b \in \Z\), we say that the integer \(m \in \Z, m \ne 0\) is the 
            least common multiple of \(a\) and \(b\) if the following holds:
            \begin{enumerate}
                \item \(\divides{a}{m}\) and \(\divides{b}{m}\), and 
                \item if \(\exists m' : \divides{a}{m'}\) and \(\divides{b}{m'}\) then \(\divides{m}{m'}\).
            \end{enumerate}
    \end{definition*}

    Recall that given \(a, b \in \Z, b > 0\), it's always possible to write
        \[
            \gcd{a, b} = \alpha a + \beta b
        \]
    for some \(\alpha, \beta \in \Z\).
        We now describe the \emph{Euclidean algorithm} to compute the GCD;
        in \emph{Section \ref{Sec:3.1}} we descride a more efficient implementation of the algorithm.
        \clearpage

    Given \(a, b \in \Z\) we proceed as follow\footnotemark:
        \begin{enumerate}
            \item Compute \(q_{i}, r_{i}\) such that \(a = q_{i}b + r_{i}\).
            \item At step \(i\), let \(a = r_{i - 1}\).
            \item Repeat until \(r_{i} = 0\).
            \item Compute \(\alpha\) and \(\beta\) by replacing \(r_{i}\) at the step \(i - 1\).
        \end{enumerate}
        \footnotetext{The procedure described assumes \(a \ge b\).}

    Most of what we will discuss later on is based on the concept of congruence modulo \(n\),
        for this reason we briefly overview what this represents.

    Let \(n \in \Z\) be a fixed integer. Then, for any \(a, b \in \Z\) 
        \[
            a \equiv b \mod n \iff \divides{n}{(a - b)}
        \]
        is an equivalence relation. In fact is:
        \begin{itemize}
            \item \emph{reflexive:} \(\forall a \in \Z, a \equiv a \mod{n} \implies \divides{n}{(a - a)} \implies \divides{n}{0}\).
            \item \emph{simmetric:} \(\forall a, b \in \Z\), if \(a \equiv b \mod{n} \implies \divides{n}{(a - b)}\),
                but then \(\divides{n}-(b - a) \implies b \equiv a \mod{n}\).
            \item \emph{transitive:} by the same reasoning done for the simmetric property.
        \end{itemize}
    Since congruence modulo n is a equivalence relation, this means we can consider 
        the equivalence classes it defines, denote these as \([a] = \set{b \in \Z}[a \equiv b \mod{n}]\).
        One can prove that congruence modulo n defines exactly \(n\) distinct classes,
        whose representatives are the integers \(0 \le k < n\).

        \begin{remark*}
            One can think of congruence modulo n, as the remainder of the division by \(n\) 
            of same \(a \in \Z\).
        \end{remark*}
\end{document}
