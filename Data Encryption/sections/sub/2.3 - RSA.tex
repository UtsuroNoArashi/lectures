\documentclass{subfiles}
\begin{document}
    The RSA algorithm, named after its inventors \emph{R. Rivest, A. Shamir \emph{and} L. Adleman},
        is a encryption algorithm based on the hard problem of integer factorization. In fact, 
        is well known that for \(n \in \Z\) sufficiently large, computing its prime factors 
        is not an easy task.

    As for the Diffie-Hellman algorithm, assume A and B to be the entities 
        involved into the communication. The algorithm proceeds as follow: 
        one of the two entities, say A, chooses two prime numbers \(p\) and 
        \(q, p \ne q\) with approximately the same length, and computes \(n = pq\)
        and \(\lambda = \lcm{p - 1, q - 1}\).
        Additionally they choose an integer \(e\) such that \(\gcd{e, \lambda} = 1\);
        that is, \(e, \lambda\) are coprime. The pair \((n, e)\) is A's public key. 
        Since \(e, \lambda\) are comprime Bezout's identity holds, therefore \(1 = ed + \lambda\mu\).
        Let \(m\) be a message B wants to send to A, they cypher it as \(c \equiv m^{d} \mod{n}\)
        and transmit \(c\) to A. On the receiving side, A takes \(c\) and decyphers it as 
        \begin{equation}\label{Eq:1}
            m = c^{d} \mod{n}
        \end{equation}

    Understanding why \emph{Equation \ref{Eq:1}} holds is not immediate, 
        for this reason the remainder of this section will be used to show that \(m \equiv c^{d} \mod{n}\)
        is indeed correct. Begin by observing that \(c \equiv m^{e} \mod{n}\), 
        meaning that \(c^{d} \equiv (m^{e})^{d} \mod{n}\).
        Recall that Bezout's identity holds, thus \(c^{d} \equiv m^{1 - \lambda\mu} \mod{n}\).
        But \(\lambda = \lcm{p - 1, q - 1}\), thus \(m^{1 - \lambda\mu} = m^{1 - (p - 1)s\mu}\).
        At this point there are two possibilities:
        \[
            c^{d} \equiv \begin{cases}
                0 \mod{p}, \text{ if } m = 0, \text{ or} \\ 
                m \mod{p}, \text{ if } m \ne 0. 
            \end{cases}
        \]
        In fact, if \(m \ne 0\), we have that
        \[\begin{aligned}
            \left\lparen m^{p - 1} \right\rparen^{s} & \equiv m^{p} \mod{p} \\ 
                & \implies m^{p - 1} \equiv 1 \mod{p} \\
                & \implies m^{\lambda} \equiv 1 \mod{p} \\ 
                & \implies m^{\lambda\mu} \equiv 1^{\mu} \mod{p}
        \end{aligned}\]
        Since a similar reasoning can be done for \(\lambda = (q -1)t\),
            we can conclude that \(c^{d} \equiv m \mod{p} \land c^{d} \equiv m \mod{q}\).
            Furthermore \(p \ne q\), therefore \(c^{d} \equiv m \mod{n}\).

        \subsubsection{Complexity of RSA}
        \subfile{../subsub/2.3.1 - Complexity of RSA}
\end{document}
