\documentclass{subfiles}
\begin{document}
    Initially posposed by \emph{A. Lempel \emph{and} J. Ziv} in \cite{lempel1976, lempel1977, lempel1978},
        these are a type of dictionary based compressors in which the dictionary is built dinamically;
        that is, instead of having a pre-defined codebook, we create it while the text is being read.
        The reason this works is simple: with an high degree of probability,
        the text previously encoded is a great soure of patterns 
        (it shares the same language, style and structure of the upcoming text).

    One of the most important aspects of LZ-based compressors
        is the fact that there's no need to transmit the dictionary, 
        as we shall see shortly.

    \begin{remark*}
        Though other variants exists, 
            in the next few sections we'll focus on LZ-77, LZss, LZ-78 and LZW.
    \end{remark*}
\end{document}
