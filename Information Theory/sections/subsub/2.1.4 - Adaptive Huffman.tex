\documentclass{subfiles}
\begin{document}
    As it was pointed out previously, 
        the main issue with the classic Huffman approach,
        is the necessity to scan the text twice,
        which slows down the encoding.
    Also, to work properly we need to know the symbols probabilities in advace,
        which is not always the case.
    To solve both these issues, a new approach has been developed known as 
        \emph{Adaptive Huffman}.

    The key point of this new approach is the fact that the tree is built and updated dynamically.
    Essentially we:
    \begin{itemize}
        \item Start with an initially empty tree,
            or one which only contains a special \textbf{Not Yet Transfered} node.
        \item For each symbol in the text: 
            \begin{itemize}
                \item If the symbol was already encoded, 
                    endode it with the existing code.
                \item If it's a new symbol:
                    encode the \textbf{NYT} node first, 
                    then encode the symbol itself and add the symbol to the tree as a leaf.
                \item After encoding a symbol, update the tree:
                    increment the frequency of the symbol and its ancestors.
                    Reorganize the tree to mantain the Huffman properties.
            \end{itemize}
    \end{itemize}

    One can prove that the adaptive approach has a better locality the the classic one.
\end{document}
