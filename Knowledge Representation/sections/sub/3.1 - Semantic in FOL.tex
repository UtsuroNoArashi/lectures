\documentclass{subfiles}
\begin{document}
    So far we have discussed the syntactic aspect of FOL, but in doing so we 
        made a semantic interpretation of the terms. To clearly untie semantic 
        and syntax we have to introduce the concepts of \emph{functional symbols} 
        and \emph{predicative symbols}, instead of functions and predicates respectively.

    To define the semantic of a given formula we must fix a domain \(D\), an 
        interpretation function \(I : D^{n} \to D\) and an interpretation of th thee predicates.
        We define this way an interpretative structure. 
        Similarly for what discussed about Propositional Logic, given close formula \(F\),
        we call \emph{model} for \(F\), a interpretative structure that makes \(F\) true.
        Again, if \(F\) admits at least a model, we say that \(F\) is satisfiable,
        otherwise is unsatisfiable. We say that \(A\) is valid, if it's true for any interpretative structure.
\end{document}
