\documentclass{subfiles}
\begin{document}
    The concept of average code length allows us to define a intresting class of 
        codes: the \emph{compact} (or optimal) codes.
        These are \gls{ud} codes that have the lowest \gls{acl}.
    
    A first attempt to achive such codes was firstly proposed by Shannon 
        and \emph{Robert Mario Fano}, who developed the so called Shannon-Fano 
        encoding (1949).

    \paragraph{Shannon-Fano encoding}
    We will be considering the binary case. 
    The encoding itself is very simple: we order the symbols in decreasing order,
    divide the symbols in two sets such that the sum of probabilities in each set 
    is almost equal, encode the symbols in the first set with 0 and the other with 1.
    Lastly, repeat the procedure for each set recursively.
    \begin{example*}
        Let's assume the alphabet \(\Set{a, b, c, d, e}\) whose probabilities are
            \(\Set{\tfrac{1}{2}, \tfrac{1}{4}, \tfrac{1}{8},
                \tfrac{1}{16}, \tfrac{1}{16}}\). 
            Applying the steps above we get the following. 
            \subfile{../../extra/TikZ/Example - SF encoding}
    \end{example*}
    One can prove that Shannon-Fano encoding is not optimal.
    \clearpage
\end{document}
