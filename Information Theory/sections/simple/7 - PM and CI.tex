\documentclass{subfiles}
\begin{document}
    Let us consider the following problem: given a document, 
        can we search for a pattern in it efficiently? 
        It is well known that several algorithms exist to solve this problem,
        the most notable of which is the \emph{Knuth-Morris-Pratt (KMP)} algorithm. 

    A natural question then is whether similar efficiency can be achieved when the 
        document is compressed; that is, whether both space and time efficiency are attainable
        in pattern matching on compressed texts. 
        An affirmative answer was provided by Ferragina and Manzini through a novel data structure
        they defined: the \emph{FM-index}.

    The remainder of this section will describe the algorithm to build the FM-index for a given text, 
        though we refer the reader to \cite{ferragina2000} for further details.
        We first discuss the \emph{Backward-search} algorithm applyied to text compressed using
        BWT-based compressors, we then discuss the results of Ferragina and Manzini.

    We recall that for any given string \(S\), the output of BWT(\(S\)) is the last colunm L 
        and the index I correspondig to the position S in the list of permutions.
        To work properly the Backward-search requires, alongside L, 
        two auxiliaries data structure:
        \begin{itemize}
            \item \lstinline{C[1, ..., n]} (n begin the size of the alphabet) that at
                \lstinline{C[c]} stores the number of characters lexicographically smaller the c in T, and 
            \item \lstinline{Occ(c, q)} stroring the occurencies of c in the prefix L[1, q].
        \end{itemize}

        Let \(P\) be the pattern we are interested in. 
        The algorithm proceeds as follow: we first read the right-most character in \(P\),
        say it's c, we then we consider \lstinline{First = C[c] + 1} and \lstinline{Last = C[c + 1]}. 
        We continue updating the value of c, \lstinline{First} and \lstinline{Last} accordingly 
        until either First > Last or we have reached the end of the pattern.
        The full algorithm is shown in \emph{Figure \ref{Fig:8}}.
        \subfile{../../extra/TikZ/Figure 8 - Backward-search algorithm}

        \begin{example*}
            Consider \(L = BTW(\omega) = ipssm\$pissii\) for some string \(\omega \in \Sigma^{*}\),
            and let \(P = pssi\). Since \(L\) is the output of the BWT of some string,
            we know how to compute the \(F\) column\footnotemark; thus we get \(F = \$ iiiimppssss\).
            We summarize the example in \emph{Figure \ref{Fig:9}}.
            \subfile{../../extra/TikZ/Figure 9 - Example of BS}

            By the algorithm in \emph{Figure \ref{Fig:8}}, we begin by considering 
            \(c = P[4] = i\) and, since \(C\) is known (we can easily compute it from \(F\)),
            we compute \(First = 2\) and \(Last = 5\). Since no halt condition is met,
            we proceed reading the next symbol in the pattern and update both \(First\) and \(Last\).
            Hence, we have \(c = P[3] = s\) and \(First = 9\) and \(Last = 10\). 
            Again, the looping conditions are met, thus we update \(c, First \text{ and } Last\)
            once again. Proceeding analogously for the reminder of \(P\), 
            we find the \(P\) does not appear in \(\omega\).

        \end{example*}
        \footnotetext{We use \(F\) to visualize what \(C\) actually represents.}

        Let us observe that the number of iteration as strictly depending on the lenght of the pattern;
            additionally it is strongly affected from the computation of \lstinline{OCC}.
            Thus, if we can construct a \lstinline{OCC} array such that \lstinline{OCC[c][q]} = \(Occ(c, q)\),
            then the backward-search would take \bigO{\abs{P}}.
            This idea allowed Ferragina and Manzini to create a compressed index -- the FM-index -- 
            by using an implementation of the backward-search that works in \bigO{\abs{P}} and requires just 
            \(5n \ \entropy{T}[k] + o(n)\) bits.

        \subsection{FM-index}\label{Sec:7.1}
        \subfile{../sub/7.1 - FM-index}
\end{document}
