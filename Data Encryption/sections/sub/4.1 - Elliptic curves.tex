\documentclass{subfiles}
\begin{document}
    \begin{definition*}[Hyper elliptic curves]
        Let \(n > 0\) be a positive integer. We call hyper elliptic curve on a field 
        \(\F\) the set of points in \(\Ps^{2}(\F)\) satisfying the equation 
        \[
            y^{2} = x^{2n + 1} + ax + b.
        \]
        In the case of \(n = 1\), we talk about elliptic curves.
    \end{definition*}

    \begin{example*}
        Let us consider the curve \(y^{2} = x^{3} - x\), and let \(\F = \R\). 
        Then, the curve we get is the following.
        \subfile{../../extra/Tikz/Figure . - Example of elliptic curve}
    \end{example*}

    We now show, given two point \(P_{1}\) and \(P_{2}\) on curve \(\mathcal{C}\),
        how to compute the sum \(P_{1} + P_{2}\). Observe that given any two distinct point,
        we can consider the line passing through them; that is, given \(P_{1} = (X_{1}, Y_{1})\) 
        and \(P_{2} = (X_{2}, Y_{2})\), the line passing both has equation 
        \begin{equation}\label{Eq:2}
            Y - Y_{1} = \frac{Y_{2} - Y_{1}}{X_{2} - X_{1}}(X - X_{1}).
        \end{equation}
        Hence, combining the equation of the curve and the one above, we can 
        easily compute the points in which these meet. We then consider the point that differ 
        from the initial ones and consider the equation of the vertical line that passes through it.
        Once again, we combine the latter and the curve equation and get the point \(R = P_{1} + P_{2}\).

    \subsubsection{ElGamal on elliptic curves}
    \subfile{../subsub/4.1.1 - Elgamal on elliptic curves}
\end{document}
