\documentclass{subfiles}
\begin{document}
    One may think that non-singular codes are enough, 
        but in most cases they are not.
    In fact, it might happen that a codeword is prefix of another,
        making the decode process tedious.
    \begin{example*}
        Let's consider the following codes.
        \subfile{../../extra/TikZ/Example - Non singular codes}
        How should we decode the string 000111? 
        Once again, we have many possibilities:
            for example \(s_{1}s_{1}s_{3}s_{2}s_{2}\) or \(s_{1}s_{1}s_{3}s_{4}\).
        Additionally, unless contex is given, we don't know which one is correct.
    \end{example*}
    We say that a code is \gls{ud} if and only each sequence of codewords
        corresponds to at most one sequence of source symbols.
    
    From this definition, two questions follow:
    \begin{itemize}
        \item How we construct such codes?
        \item How can we check whether a given code is \gls{ud} or not?
    \end{itemize}

    The next section focus on the second question, we'll then answer the first.
\end{document}
