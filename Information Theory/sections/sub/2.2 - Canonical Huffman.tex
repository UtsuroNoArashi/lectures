\documentclass{subfiles}
\begin{document}
    Let us consider some limitations of the Huffman encoding discussed so far.
    One of the main issues 
        is that the encoder must transmit, alongside the text,
        the tree used for encoding, which constitutes a non-trivial overhead.
    Additionally, 
        the decoding phase is slow due to the necessity of traversing the tree for each symbol.

    It should be noted that there exist compact codes
        that are not produced by the Huffman algorithm.
    We now introduce the so-called \emph{Canonical Huffman encoding},
        a variant of Huffman that addresses the issues above by requiring the 
        encoder to transmit only the lengths of the codewords.

    This encoding is particularly useful when the source alphabet is large
        and fast decoding is required.

    \subsubsection{The encoding}
    \subfile{../subsub/2.2.1 - Canonical Huffman encoding}

    \subsubsection{The decoding}
    \subfile{../subsub/2.2.2 - Canonical Huffman decoding}
\end{document}
