\documentclass{subfiles}
\begin{document}
    To begin with, let us consider some string \(\omega \in \Sigma^{*}\) for some 
        alphabet \(\Sigma\). When would we define \(\omega\) to be random?
        Intuitively, when no pattern shows with it. Thus,
        if we would like to describe \(\omega\), we should use \(\omega\) itself,
        because this is the best we can do; that is, no simpler description exists.

    This same idea is at the base of what is known as \emph{Kolmogorov complexity};
        i.e., the shortest computer program need to describe some mathematician object.
        More rigorousely, fixed some model of computation, the Turing machine in our case,
        the Kolmogorov complexity of a string \(S\) is defined as 
        \[
            K_{U}(S) = \min \set{l(p)}[U(p) = S]
        \]
        where \(p\) is a program, \(l(p)\) its length and \(U(p)\) the string produced 
        by the TM \(U\) when runnin \(p\).

    We now provide two theorems related to Kolmogorov complexity.
    \begin{theorem}\label{Thm:8.1}
        Let \(U, V\) be two universal Turing machines. For any string \(x\)
        it holds that \(\abs{K_{U}(x) - K_{V}(x)} \le C(U, V)\) where \(C(U, V)\)
        depends only on \(U\) and \(V\).
    \end{theorem}

    \begin{theorem}\label{Thm:8.2}
        For any arbitrary string \(x\), \(K(x)\) is not computable.
    \end{theorem}

    \emph{Theorem \ref{Thm:8.1}} tells us that Kolmogorov complexity is invariant,
        while the proof of \emph{Theorem \ref{Thm:8.2}} follows directly from the relationship 
        between G\"odel's incompleteness theorems and the halting problem for a Turing machine.



\end{document}
