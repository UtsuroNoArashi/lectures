\documentclass{subfiles}
\begin{document}
    Pollard's \(\rho\) algorithm, proceeds to define a sequence 
        \(\set{x_{i}}_{i \in I}, x_{i} \in G = \Z_{p}\) and it stops when 
        we find \(x_{i}, x_{j}\) such that \(x_{i} \equiv x_{j} \mod{p}\).
        The question then is: how do we define this sequence?
        We first begin by partioning \(G\) into three sets \(S_{0}, S_{1}\) 
        and \(S_{2}\) of roughly equal size. For instance, let 
        \[
            S_{i} = \set{x \in G}[x \equiv i \mod{3}], i = 0, 1, 2.
        \]
        Then, define the sequence as follow:
        \[
            x_{i + 1} = \begin{cases}
                \alpha x_{i}, \quad \text{if } x_{i} \in S_{0} \\ 
                x_{i}^{2}, \quad \text{if } x_{i} \in S_{2} \\ 
                \gamma x_{i}, \quad \text{if } x_{i} \in S_{1}
            \end{cases}
        \]
        for \(i \ge 0\), with \(x_{0} = 1\).
        Observe that any \(x_{i}\) can be expressed as the product \(\alpha^{a_{i}}\gamma^{b_{i}}\),
        for some \(a_{i}, b_{i}\). Thus, it holds 
        \[
            (b_{i} - b_{j}) \cdot \log_{\alpha} \gamma \equiv (a_{i} - a_{j}) \mod{n}.
        \]
        Provided that \(b_{i} \ne b_{j}\) (the case \(b_{i} = b_{j}\) happens 
        with a negligible probability), the discrete logarithm can be easily computed.
        It can be proved that Pollard's \(\rho\) algorithm takes \bigO{\sqrt{n}} time.
\end{document}
