\documentclass{subfiles}
\begin{document}
    Proposed for the first time in \cite{diffieH1976}, it was a breaktrhough in 
        cryptography. The idea is that of crypting the message as some 
        discrete value, by means of a one-way function; that is, 
        a function that is easy to compute, but whose inverse is hard to compute.
        
    The whole cryptosystem relies on the assumption that one-way functions do exists;
        through the years several have been proposed and are used in schemes we discuss 
        in the following, yet these are not garanteed to be one-way\footnotemark.

    \footnotetext{If one proves the existence of one-way functions, 
        at the same time would solve the \(\P* = \N*P\) problem.}
\end{document}
