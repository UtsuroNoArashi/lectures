\documentclass{subfiles}
\begin{document}
    We consider the binary case; the extension to the general case is straightforward.
        Let 
        \[
            S = \begin{pmatrix}
                s_{1} & s_{2} & \cdots & s_{n} \\ 
                p_{1} & p_{2} & \cdots & p_{n} 
            \end{pmatrix}
        \]
        denote a source with probabilities arranged in non-increasing order.
    
    Define \(R(S)\) as the \emph{reduced source}, obtained by replacing the two least probable
        symbols in \(S\) with a single symbol whose probability is the sum of the merged symbols:
        \[
            R(S) = \begin{pmatrix}
                s_{1} & s_{2} & \cdots & (s_{n - 1}, s_{n}) \\ 
                p_{1} & p_{2} & \cdots & p_{n - 1} + p_{n} \\
            \end{pmatrix}
        \]

    Let \(\C*_{R}\) denote a binary prefix code for \(R(S)\), and let \(z\) be the codeword assigned to 
        the merged symbol \((s_{n - 1}, s_{n})\).
    A prefix code \(\C*\) for \(S\) can then be obtained from \(\C*_{R}\) by assigning 
        the \(i\)-th symbol of \(S\) the \(i\)-th codeword in \(\C*_{R}\) for \(i \le n - 2\). 
        The codewords for \(s_{n - 1}\) and \(s_{n}\) are \(z0\) and \(z1\), respectively.

    \clearpage 
    \subsubsection{The encoding}
    \subfile{../subsub/2.1.1 - Huffman encoding}

    \subsubsection{The decoding}
    \subfile{../subsub/2.1.2 - Huffman decoding}

    \subsubsection{Adaptive Huffman}
    \subfile{../subsub/2.1.3 - Adaptive Huffman}
\end{document}
