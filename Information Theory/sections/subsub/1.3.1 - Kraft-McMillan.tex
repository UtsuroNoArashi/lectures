\documentclass{subfiles}
\begin{document}
    We have disscussed what prefix-free codes are, 
        and what the average code length represents.
    We now provide a necessary and sufficient condition for the existence of 
    a prefix code: the \emph{Kraft-McMillan inequality}.
    \begin{theorem}[Kraft-McMillan]\label{Thm:2}
        Let us consider a source alphabet \(S = \Set{s_{1}, \ldots, s_{n}}\)
        and a code alphabet \(X =\Set{x_{1}, \ldots, x_{d}}\).
        Let \(l_{1}, \ldots, l_{n}\) be a set of lengths.
        Then necessary and sufficient condition for the existence of a 
        prefix-free code \(C\) over the alphabet \(X\) with codewords lenghts 
        \(l_{1}, \ldots, l_{n}\) is that 
        \[
            \sum_{i = 1}^{n}{d^{-l_{i}}} \le 1
        \]
        with \(d\) size of the code alphabet.
    \end{theorem}
    In other terms, if a set of lengths satisfies the inequality then,
        there exist at least a way to arrange the codewords into a prefix code.
    A proof of this theorem is beyond the scope of this \underline{\em notes}.
\end{document}
