\documentclass{subfiles}
\begin{document}
    \begin{definition*}[Ring]
        Let \(R\) be a a set, and let \(+, \cdot\) be two binary operations on \(R\).
            We say that the triplet \((R, +, \cdot)\), often as \(R\), is a ring if and only if 
            the following conditions hold.
            \begin{enumerate}
                \item For all \(a, b \in R, a + b = b + a\).
                \item \(a + (b + c) = (a + b) + c, \forall a, b, c, \in R\).
                \item There exist a neutral element for \(+\); that is,
                    \(\exists 0 \in R \text{ s.t. } a + 0 = a, \forall a \in R\).
                \item Each element has an additive oppposite, i.e., \(\forall a \in R,
                    \exists -a \in R\) such that \(a + (-a) = (-a) + a = 0\).
                \item \(\cdot\) is associative; that is, \(a(bc) = (ab)c, \forall a, b, c \in R\).
                \item For all \(a, b, c \in R, a(b + c) = ab + ac\).
            \end{enumerate}
    \end{definition*}

    \begin{definition*}[Unital ring]
        Let \((R, +, \cdot)\) be a ring. We say that \(R\) is a unital ring, 
            or a ring with unity, if there exists \(1 \in R\) such that \(1 \ne 0\)
            and \(1a = a1 = a, \forall a \in R\).
    \end{definition*}

    \begin{definition*}[Abelian ring]
        Let \((R, +, \cdot)\) be a ring. If for all \(a, b \in R\) it holds that 
            \(ab = ba; R\) is called commutative, or abelian, ring. 
    \end{definition*}

    \begin{definition*}[Integral domain]
        Let \((R, +, \cdot)\) be a commutative ring. If, for all \(a, b \in R\)
            such that \(ab = 0\), either \(a = 0\) or \(b = 0\); 
            we say that \(R\) is an integral domain.
    \end{definition*}

    \begin{definition*}[Division ring]
        Let \((R, +, \cdot)\) be a unital ring. We say that \(R\) is a division
            ring if, for all \(a \in R\) exists a unique \(a^{-1} \in R\) such that 
            \(aa^{-1} = a^{-1}a = 1\).
    \end{definition*}

    \begin{definition*}
        Let \((R, +, \cdot)\) be a ring. We say that \(R\) is a field, 
            if and only if  \(R\) is a commutative division ring.
    \end{definition*}

    In the following we denote fields with \(\F\), while Galois fields, 
        i.e., fields with finitely many elements, with \(\gf{q}\) where \(q\)
        is the number of elements in the fields.

    In \emph{Section \ref{Sec:5}} we discuss several codes, most of which make use of polynomial.
        In this context, an issue may occur: the chosen polynomial has no solution within the field in use.
        To solve it, we make use of field extensions, which are formally defined below.

    \begin{definition*}
        Let \(\F\) be a field. Denote by \(\F[x]\) the set of polynomials with coefficients
            in \(\F\)\footnotemark[1]. Let \(f(x) \in \F[x]\) be a irreducible polynomial,
            i.e., a polynomial with no solution\footnotemark[2]. We call field extension 
            a field \(\F'[x]\) in which we impose the existence of a solution for \(f(x)\).
            That is, 
            \[
                \F'[x] = \set{a + bi}[a, b \in \F \text{ and } i \text{ root for } f(x)].
            \]
    \end{definition*}

    \footnotetext[1]{It can be proved that \(\F[x]\) is also a field.}
    \footnotetext[2]{To be more precise a polynomial with no solution is garanteed to be irreducible
        if its degree is at most 3.}
\end{document}
