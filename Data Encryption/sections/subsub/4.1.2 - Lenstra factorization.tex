\documentclass{subfiles}
\begin{document}
    Lenstra factorization algorithm, also knows as \gls{ecm}, 
        is a factorization algorithm for integers by means of elliptic curves.
        In it's essence, the algorithm is very simple, as we are about to describe.

    Start by considering an elliptic curve \(\mathcal{C}\),
        in Weierstrass form (eg: \(y^{2} = x^{3} + 2x + 3\) over \(\gf{7}\)), 
        and a point \(P \in \mathcal{C}\). Since, in general we consider \(\Z_{n}\),
        we will be talking about a ring.

    Fix a upperbound \(B\), as in the case of Pollard factorization, and 
        compute \(2P, 3P, \ldots, kP, \ldots\) up to \(B\).
        At this point two things may happen:
        \begin{enumerate}
            \item The computation of all \(B - 1\) points proceeds flawlessly;
                in which case we can either attempt with a different curve, a 
                different point or both, or assume that the number we are trying
                to factor is actually a prime number\footnotemark. 
            \item At some point during the computation, we fail. That is, 
                we have found some element \(v = (X_{2} - X_{2})\) (\emph{Equation \ref{Eq:2}})
                which has no inverse. Hence, a non trivial divisor is given by 
                \(\gcd{n, v}\).
    
        \end{enumerate}

    About the complexity: ECM is sub-exponential, that is, it's at the edge between 
        polynomial and exponential algorithms.  More rigorously, ECM has a complexity of 
        \(\subExp{\sfrac{1}{2}}{1}{p}\), where
        \[
            \subExp{\alpha}{c}{x} = \exp^{((c + \bigO{1})(\ln x)^{\alpha}(\ln \ln x)^{1 - \alpha})}
        \]
        with \(0 \le \alpha \le 1\).

        \footnotetext{More correctly, we should call these pseudoprime numbers,
        since no factorization has been found nor we are sure it's prime.}    
\end{document}
