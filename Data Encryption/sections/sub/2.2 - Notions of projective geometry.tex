\documentclass{subfiles}
\begin{document}
    In what follows we assume that notions such as points, lines,
        planes in the geometrical sense, in addition to the notion of vector spaces
        are known to the reader.

    \begin{definition*}[Ray]
        Let \(\F\) be a field, and let \(\V\) be a vector space over \(\F\).
            Let \(\mathbf{v} \in \V\) be non-zero. We call the set 
            \[
                [\mathbf{v}] = \set{k \mathbf{v}}[k \in \F^{*}]
            \]
            a ray.
    \end{definition*}

    \begin{definition*}[Projective plane]
        Let \(\F\) be a field and let \(\V\) be a vector space over \(F\).
            Let \(\dim{\V} = 3\). We define the projective plane \(\P^{2}\) as 
            the set of rays of \(V\). An element \(X \in \P^{2}\) is called a point.
    \end{definition*}

    In our discussion we usually consider projective points in \emph{homogeneous}
        coordinates; that is, if \(X = [x : y : z]\) is a projective point,
        at least one of \(x, y, z\) is non-zero.

    The relation with ordinary plane geometry is done as follow:
        \begin{itemize}
            \item Any point \((x, y) \in \R^{2}\) becomes a projective point 
                \([x : y : 1]\); and 
            \item Any point \([x : y : 0] \in \P^{2}\), which do not belong to 
                Euclidean geometry, are called point at infinity, and 
                we denote them as \(\mathcal{O}\).
        \end{itemize}

    We now present \emph{elliptic curves} which are the main reason we discuss 
        projective geometry.

    \begin{definition*}[Elliptic curves]
        Let \(\V\) be a vector space over a field \(F\). 
            An elliptic curve \(\C*\) is the set of points of \(\P^{2}\),
            that are solution to an equation of the form
            \[
                y^{2} = x^{3} + ax + b
            \]
            with \(a, b \in \F\).
    \end{definition*}
    
    In our treating, we consider non-singular elliptic curves; that is, 
        such that the discriminant is non zero, i.e., \(\Delta = -16(4a^{3} + 27b \ne 0)\).

    In \emph{Section \ref{Sec:3.3.2}} we treat the ElGamal algorithm by means of 
        elliptic curves; in this context given \(\C*\) an elliptic curve,
        we are required to compute the sum of two point \(P, Q \in \C*\),
        which is non trivial. In what remain of this section, we shall present how 
        to do such addition.

    Let \(\C*\) be an ellitptic over the field \(\F\) and let \(P = (x_{1}, y_{1}),
    Q = (x_{2}, y_{2}) \in \C*\).
        Then, the following cases hold:
        \begin{itemize}
            \item The points are opposite of one another, 
                then \(P + Q = P + \mathcal{O} = P\).

            \item The two points are distinct; then we consider the line that passes 
                through them. That is, we compute 
                \[
                    x_{3} = m^{2} - x_{1} - x_{2} \qquad y_{3} = m(x_{1} - x_{3}) - y_{1}
                \]
                where \(m = \sfrac{(y_{2} - y_{1})}{(x_{2} - x_{1})}\) is the slope.

            \item The two point coincide; then we consider the tangent line.
                That is, we compute 
                \[
                    m = \frac{3x_{1}^{2} + a}{2y_{1}}
                \]
                and \(x_{3}, y_{3}\) as before.
        \end{itemize}
\end{document}
