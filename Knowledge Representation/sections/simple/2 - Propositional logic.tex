\documentclass{subfiles}
\begin{document}
    At the core of Propositional Logic are atomic proposition, 
        that are, any simple assertion. This means that, questions or
        meaningless sentences do not correspond to a proposition.

    To build up more complex proposition we make use of connectives,
        that, up to a degree, have the same meaning to the English counterparts.
        Such connectives are: \(\lnot\) (not) and \(\land\) (and).
        From these two, three additional connectives can be derived, 
        though the latter are more a shorthand. Such additional connectives are:
        \(\lor\) (or), which is equivalent to \(\lnot(\lnot p \land \lnot q)\),
        \(\implies\) (implies), equivalent to \(\lnot p \lor q\)
        and \(\iff\) (if and only if), which corresponds to 
        \(p \implies q \land q \implies p\), where \(p\) and \(q\) are atomic propositions.
        We use the symbol \textbf{1} and \textbf{0}, to indicate, respectively,
        a true and a false statement.

    \subsection{Validity, satisfiability and contradiction}
    \subfile{../sub/2.1 - Validity of PL}

    \subsection{Consequence and equivalence}
    \subfile{../sub/2.2 - Consequence and equivalence}

    \subsection{Rules, soundness and completeness of propositional logic}
    \subfile{../sub/2.3 - Rules of propositional logic}

    \subsection{Horn formulas and resolution}
    \subfile{../sub/2.4 - Horn formulas and resolution}
    \clearpage
\end{document}
