\documentclass{subfiles}
\begin{document}
    When we talk about Elias code we refer, generally, 
        to two codes (the \emph{gamma} code and the \emph{delta} code),
        proposed by \emph{Peter Elias}\footnotemark[2] in \cite{Elias}.

    Before we start talking about the codes,
        let us define some notation useful to understand what follows.
        Let \(x \text{ be an Integer, and let } B(x)\) be its binary representation.
        We define \(\Abs{B(x)}\) as the number of bits needed to represent \(x\).
        \note{One can also think of \(\Abs{B(x)}\) as the length of \(B(x)\)}

        \footnotetext[2]{He was a colleague of Shannon's, and also one of Fano's students.}

    \subsubsection{Gamma code}
    \subfile{../subsub/4.2.1 - Gamma code}

    \subsubsection{Delta code}
    \subfile{../subsub/4.2.2 - Delta code}
\end{document}
