\documentclass{subfiles}
\begin{document}
    In this section we show how the resolution algorithm presented in \emph{
        Section \ref{Sec:2.4.1}} can be extended to formulas in FOL. 
        As done in the case of Propositional Logic, we need to convert each formula 
        into its clause form. But as we are about to show, this is not so straightforward.

    The procedure we are about to show is not unique; apart from the last two steps,
        all other steps can be performed in any order. Since in general we deal with 
        long formulas, we suggest as first step to remove any material condition, i.e.,
        implications. Then move each negation inward, that is, replace each 
        \[
            \lnot \forall x \ P(x) \text{ with } \exists x \ \lnot P(x)
        \]
        and
        \[
            \lnot \exists x \ P(x) \text{ with } \forall x \ \lnot P(x).
        \]

    As third step, if needed, to avoid confusion, standardize variable apart, 
        meaning that, if the same variable is bounded to two, or more, quantifiers,
        one should introduce a new variables. For instance, take 
        \[
            \forall x \ \lnot Q(x, z) \lor \exists x \ P(x),
        \]
        here the second \(x\) can be replaced by the variable \(y\) to make more clear 
        the distinction of the two parts; hence the formula becomes 
        \[
            \forall x \ \lnot Q(x, z) \lor \exists y \ P(y).
        \]

    As next step we introduce the \gls{pnf}, that is, a formula in the form 
    \[
        Q_{1} x_{1} \ Q_{2} x_{2} \ \cdots \ Q_{n} x_{n} \ P 
    \]
    where each \(Q_{i}\) is a quantifier and \(P\) is without quantifiers.
    For instance: \(\exists x \ \lnot P(x) \lor \exists y \ Q(y)\) becomes 
    \(\exists x \ \exists y \ \lnot P(x) \lor Q(y)\). We then transform the formula in PNF 
    to its \gls{snf}, i.e., we remove all existential quantifiers. More specifically,
    each quantifiers of the form \(\exists y\) is replaced by some functional symbol 
    \(f(x_{1}, \ldots, x_{n})\) where each \(x_{i}\) is a universal quantified variable 
    preceding \(y\), or a constant if none. Lastly, we remove all universal quantifiers
    and assume the formula as universally quantified. 

    Note that formulas in SNF are not equivalent to the original ones; 
        despite this, thaks to the following theorem and the fact that the resolution
        algorithm is refutation-complete, the problem does not exitst.

    \begin{theorem*}[Skolem's theorem]
        A formula is unsatisfiable if and only if the SNF of its universal 
        closure is unsatisfiable.
    \end{theorem*}
    
    Once the formula is in SNF the conversion to CNF and clause form is straightforward.

    \subsubsection{The resolution rule}
    \subfile{../subsub/3.2.1 - The resolution rule}
\end{document}
