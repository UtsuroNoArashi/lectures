\documentclass{subfiles}
\begin{document}
    \begin{definition*}[B-smoothness]
        Let \(B\) be a positive integer. We say that \(n \in \Z\) is B-smooth if 
        all its prime factor are less or equal to \(B\). That is, if 
        \[
            \divides{p_{i}}{n} \implies p_{i} \le B.
        \]
    \end{definition*}

    Pollard's \(p - 1\) factorization is based on the following:
        fix a bound \(B\). Define \(Q\) as the LCM of all primes powers less or equal \(B\) 
        that are less or equal to \(n\). Observe that if \(q^{l} \le n\), then \(l \log q \le \log n\),
        thus \(l = \floor{\tfrac{\log n}{\log q}}\).
        Therefore
        \[
            Q = \prod_{q \le B} q^{\frac{\log n}{\log q}}.
        \]

    Note that \(q \le p - 1 < p \le n\), hence if \(p\) is a prime factor of \(n\) 
        and \(p - 1\) is \(B\)-smooth, then \(\divides{p - 1}{Q}\);
        consequently for any \(a\) such that \(\gcd{a, p} = 1\), by Fermat's little theorem 
        we have \(a^{Q} \equiv 1 \mod{p}\). For this reason, if \(d = \gcd{a^{Q} - 1, n}\),
        then \(\divides{p}{d}\).

    Below we provide the pseudo-code to implement Pollard's \(p - 1\) factorization.
    \subfile{../../extra/Tikz/Figure . - Pseudocode of Pollard p-1 factorization}
\end{document}
