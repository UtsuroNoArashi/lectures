\documentclass{subfiles}
\begin{document}
    Consider \(\Z\), and let \(a, b \in \Z, b \ne 0\). 
        We say that \(a\) divides \(b\), and write \(\divides{a}{b}\), 
        if exists \(q \in \Z\) such that \(aq = b\). Viceversa,
        we say that \(a\) does not divide \(b\) and we write \(\ndivides{a}{b}\).

    From the above notion of divisibility, we derive the notion of \gls{gcd} and \gls{lcm} of two numbers.
    \begin{definition*}[GCD]
        Let \(a, b \in \Z\), we say that the integer \(d \in \Z, d > 0\) is the 
            greatest common divisor of \(a\) and \(b\) if the following holds:
            \begin{enumerate}
                \item \(\divides{d}{a}\) and \(\divides{d}{b}\), and 
                \item if \(\exists d' : \divides{d'}{a}\) and \(\divides{d'}{b}\) then \(\divides{d'}{d}\).
            \end{enumerate}
    \end{definition*}

    \begin{definition*}[lcm]
        Let \(a, b \in \Z\), we say that the integer \(m \in \Z, m \ne 0\) is the 
            least common multiple of \(a\) and \(b\) if the following holds:
            \begin{enumerate}
                \item \(\divides{a}{m}\) and \(\divides{b}{m}\), and 
                \item if \(\exists m' : \divides{a}{m'}\) and \(\divides{b}{m'}\) then \(\divides{m}{m'}\).
            \end{enumerate}
    \end{definition*}

    To compute the GCD we use the \emph{euclid algorithm}, which is based on the following remark. 
        Given \(a, b \in \Z, b > 0\), it's always possible to write \(\gcd{a, b} = \alpha a + \beta b,
        \alpha, \beta \in \Z\).
        Back to the algorithm, given \(a, b \in \Z\) we proceed as follow\footnotemark:
        \begin{enumerate}
            \item Compute \(q_{i}, r_{i}\) such that \(a = q_{i}b + r_{i}\).
            \item At step \(i\), let \(a = r_{i - 1}\).
            \item Repeat until \(r_{i} = 0\).
            \item Compute \(\alpha\) and \(\beta\) by replacing \(r_{i}\) at the step \(i - 1\).
        \end{enumerate}
        In \emph{Section \ref{Sec:3.1}} we show how 
        euclid algorithm can be enhanced to be computationally more effient.

        \footnotetext{The procedure described assumes \(a \ge b\).}

    Let \(n \in \Z\) be a fixed integer. Then, for any \(a, b \in \Z\) 
        \[
            a \equiv b \mod n \iff \divides{n}{(a - b)}
        \]
        is an equivalence relation. In fact is:
        \begin{itemize}
            \item \emph{reflexive:} \(\forall a \in \Z, a \equiv a \mod{n} \implies \divides{n}{(a - a)} \implies \divides{n}{0}\).
            \item \emph{simmetric:} \(\forall a, b \in \Z\), if \(a \equiv b \mod{n} \implies \divides{n}{(a - b)}\),
                but then \(\divides{n}-(b - a) \implies b \equiv a \mod{n}\).
            \item \emph{transitive:} by the same reasoning done for the simmetric property.
        \end{itemize}
    Since congruence modulo n is a equivalence relation, this means we can consider 
        the equivalence classes it defines, denote these as \([a] = \set{b \in \Z}[a \equiv b \mod{n}]\).
        One can prove that congruence modulo n defines exactly \(n\) distinct classes,
        whose representatives are the integers \(0 \le k < n\).

        \begin{remark*}
            One can think of congruence modulo n, as the remainder of the division by \(n\) 
            of same \(a \in \Z\).
        \end{remark*}

    We now define two of the most important algebraic structures: groups and rings;
        we point out that many more things should be said about these structures,
        but that's beyond the scope of our discussion.

    \begin{definition*}[Group]
        A \emph{group} is a pair \((G, \cdot)\) where \(G\) is a set and 
            \[
                \cdot : G \times G \to G
            \]
            is a binary operation satisfying:
        \begin{enumerate}
            \item (Associativity) For all \(a,b,c\in G\) we have \((a \cdot b) 
                \cdot c = a \cdot (b \cdot c)\).
            \item (Identity) There exists an element \(e\in G\) such that for all 
                \(a \in G\) we have \(e \cdot a = a \cdot e = a\).
            \item (Inverses) For every \(a\in G\) there exists \(a^{-1} \in G\) with
                \(a \cdot a^{-1} = a^{-1} \cdot a = e\).
        \end{enumerate}
        If, in addition, \(a\cdot b=b\cdot a\) for all \(a,b\in G\), 
            the group is called \emph{abelian} (or commutative).
    \end{definition*}

    \begin{definition*}[Cyclic group]
        A group \((G, \cdot)\) is \emph{cyclic} if there exists an element 
        \(g\in G\) such that 
            \[
                G=\set{g^n}[n \in \Z].
            \]
        Such an element \(g\) is called a \emph{generator} of \(G\). 
    \end{definition*}

    \begin{definition*}[Ring]
        A (associative) \emph{ring} is a triple \((R, +, \cdot)\) where \((R,+)\) is an abelian 
            group with identity \(0\), and \(\cdot\) is a binary operation on \(R\) satisfying:
        \begin{enumerate}
            \item (Associativity of multiplication) \(a(bc)=(ab)c\) for all \(a, b, c \in R\).
            \item (Distributivity) \(a (b + c) = ab + ac\) and \((a + b) c = ac + bc\) for all \(a, b, c \in R\).
        \end{enumerate}
        A ring is called \emph{commutative} if \(ab=ba\) for all \(a, b \in R\). 
            A ring has \emph{unity} (or \(1\)) if there exists \(1 \in R\) with 
            \(1 \cdot a = a \cdot 1 = a\) for all \(a \in R\).
        \end{definition*}

    \begin{definition*}[Field]
        A \emph{field} is a commutative ring \((\F, +, \cdot)\) with unity \(1 \neq 0\) 
            such that every nonzero element has a multiplicative inverse. Equivalently,
        \[
            (\F \setminus \set{0}, \cdot)
        \]
        is an abelian group.
    \end{definition*}

    \begin{definition*}[Finite field / Galois field]
        A \emph{finite field}, or \emph{Galois field}, is a field with finitely many elements.
    \end{definition*}
    We will consider multiplicative Galois fields of order a prime, which we denote as \(\gf{p}\).
\end{document}
