\documentclass{subfiles}
\begin{document}
    Let's look back at the example of \emph{Section \ref{Sec:1.2.1}}. 
    Given the codewords \(\set{0, 1, 01, 11}\) for the source symbol \(s_{1},
    s_{2}, s_{3}, s_{4}\) respectively, how should we decode the string \(000111\)?
    As said before, unless some context is given, we can't be sure.

    Then what's the issue? 
    Essentially, even though to each source symbol is assigned a distinct codeword,
    these are not prefix-free, meaning that some of the codewords are prefix of others
    (eg. the codeword for \(s_{1}\) is prefix of the codeword for \(s_{3}\)).
    From what just stated, a natural solution to the problem is to define codes 
    that are prefix-free.
        One can also prove that, prefix-free (or simply prefix) codes are also instataneous (each 
    source symbol can be decoded without reading further in the string).
    \begin{theorem*}
        Let \(C\) be a prefix-free code, then \(C\) is \gls{ud}.
    \end{theorem*}
    \begin{proof*}
        It follows from \emph{Theorem \ref{Thm:1}}.
    \end{proof*}
\end{document}
