\documentclass{subfiles}
\begin{document}
    ElGamal algorithm on elliptic curves, apart from some minor differences,
        works exacly the same as the one described in \emph{Section \ref{Sec:2.2}}.
        In fact, a TTP chooses an elliptic curve \(\mathcal{C}\) over the field \(\gf{p}, p\) a prime,
        and selects \(\gamma\) such that \(\gamma \in \mathcal{C}\). 
        Let \(A\) and \(B\) be the two entities that want to communicate.
        As for the algorithm described previously, \(A\) and \(B\) computes 
        \(\alpha = \gamma^{a}, \beta = \gamma^{b}\) for some private value \(a \text{ and } b\).
        Let us note that, in the case of elliptic curves, the group is additive; 
        that is, we compute \(a\gamma\) (\(b\gamma\) respectively), i.e., the sum 
        of \(\gamma\) with itself \(a\)-times (\(b\)-times respectively).
    
    Since \emph{Equation \ref{Eq:2}} suffices the sum of two distinct point, 
        we can't use it. for this reason, in general \emph{Equation \ref{Eq:2}} is replaced 
        by the equation for the tangen line.

    The remainder of the algorithm is the same as the one described in \emph{Section \ref{Sec:2.2}}.
\end{document}
