\documentclass{subfiles}
\begin{document}
    We treat the binary case; the extension to the general case is immediate.
    Let 
    \[
        S = \begin{pmatrix}
            s_{1} & s_{2} & \cdots & s_{n} \\ 
            p_{1} & p_{2} & \cdots & p_{n} 
        \end{pmatrix}
    \]
    be a source, such that the probabilities are sorted non-increasingly.
    Denote by \(R(S)\) the \emph{reduced soource}, 
        obtained by replacing the two least common symbols from \(S\),
        with one whose probabily is the sum of those of the replaced symbols.
        This means 
        \[
            R(S) = \begin{pmatrix}
                s_{1} & s_{2} & \cdots & (s_{n - 1}, s_{n}) \\ 
                p_{1} & p_{2} & \cdots & p_{n - 1} + p_{n} \\
            \end{pmatrix}
        \]
    Assume \(C_{R}\) to be a binary prefix code for \(R(S)\),
        and let \(z\) be the codeword associated to the merged symbols \((s_{n - 1}, s_{n})\).
    Then, a prefix code \(C\) for \(S\) can be obtained by \(C_{R}\) by assigning 
        to the i-th symbol of \(S\) the i-th codeword in \(C_{R}\),
        for \(i \le n - 2\). 
        The codeword for \(s_{n - 1} \text{ and } s_{n}\) are simply 
        \(z0 \text{ and } z1\) respectively.

    \clearpage 
    \subsubsection{The encoding}
    \subfile{../subsub/2.1.1 - Huffman encoding}

    \subsubsection{The decoding}
    \subfile{../subsub/2.1.2 - Huffman decoding}

    \subsubsection{Exercises}
    \subfile{../subsub/2.1.3 - Huffman exercises}

    \subsubsection{Adaptive Huffman}
    \subfile{../subsub/2.1.4 - Adaptive Huffman}
\end{document}
