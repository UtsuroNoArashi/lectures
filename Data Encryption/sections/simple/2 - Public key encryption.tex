\documentclass{subfiles}
\begin{document}\label{Sec:2}
    The algorithms we are about to discuss are based on some properties of groups 
        we did not discuss in the previous section. We did so because these are 
        somewhat obvious and can be understood to be true without a formal proof.

    In addition, when we analyze the complexity of these algorithms,
        \(n\) is the number of digits needed to represents a given integer
        in a choosen notation. This means that for each \(k \in \Z^{+}, n = \floor{\log k} + 1\).
        With this notation, addition has a complexity of \(\bigO{n}\) time, while multiplication,
        as well as division, has a complexity of \(\bigO{n^{2}}\) time.

    \subsection{Diffie-Hellman}
    \subfile{../sub/2.1 - DH}

    \subsection{ElGamal}
    \subfile{../sub/2.2 - Elgamal}

    \subsection{RSA}
    \subfile{../sub/2.3 - RSA}
    \clearpage
\end{document}
