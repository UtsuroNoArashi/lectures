\documentclass{subfiles}
\begin{document}
    We have discussed what prefix-free codes are, and what the average code length represents.
        We now provide a necessary and sufficient condition for the existence of a prefix code:
        the \emph{Kraft-McMillan inequality}.

    \begin{theorem}[Kraft-McMillan]\label{Thm:2}
        Let \(S = \set{s_{1}, \ldots, s_{n}}\) be a source alphabet and \(X = \set{x_{1}, \ldots, x_{d}}\) a code alphabet.
            Let \(l_{1}, \ldots, l_{n}\) be a set of lengths.
            Then, a necessary and sufficient condition for the existence of a 
        prefix-free code \(\C*\) over the alphabet \(X\) with codeword lengths 
        \(l_{1}, \ldots, l_{n}\) is that 
        \[
            \sum_{i = 1}^{n}{d^{-l_{i}}} \le 1,
        \]
        where \(d\) is the size of the code alphabet.
    \end{theorem}

    In other words, if a set of lengths satisfies the inequality, then
        there exists at least one way to arrange the codewords into a prefix code.
\end{document}
