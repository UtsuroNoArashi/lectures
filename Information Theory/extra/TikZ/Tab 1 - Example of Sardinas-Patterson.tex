\documentclass{subfiles}
\begin{document}
    \begin{table}[!hb]
        \centering
        \begin{tabular}{|c|c|c|c|c|c|}
            \hline 
            \(S_{0}\) & \(S_{1}\)               & \(S_{2}\)               & \(S_{3}\)               & \(S_{4}\)             & \(S_{5}\)                 \\ 
            \hline
            a       &                           &                         &                         &                       &                           \\
            c       &                           &                         & \textcolor{rpRose}{de}  &                       &                           \\ 
            ad      & \textcolor{rpLove}{d}     &                         &                         &                       &                           \\ 
            abb     & \textcolor{rpLove}{bb}    &                         &                         &                       &                           \\ 
            bad     &                           &                         &                         &                       & \textcolor{rpFoam}{ab}    \\ 
            deb     &                           & \textcolor{rpGold}{eb}  &                         & \textcolor{rpPine}{b} &                           \\ 
            bbcde   &                           & \textcolor{rpGold}{cde} &                         &                       & \textcolor{rpFoam}{bcde}  \\
            \hline
        \end{tabular}
        \caption{Steps of Sardinas-Patterson for the code \(C\). 
        Note that each \(\omega\) has been added to the row of the value use for \(\alpha\).}
        \label{Tab:1}
    \end{table}    
\end{document}
