\documentclass{subfiles}
\begin{document}
    Let us consider one of the most notorious problems in biology: that of CG-islands.
        The problem is the following: is well known that each neucleotide 
        has probabily \(\sfrac{1}{4}\) of occurring within a gene;
        hence, any dineucleotide should appear with a frequency of \(\sfrac{1}{16}\).
        It turns out that this is not the case, in fact the CG dineucleotide is 
        underrepresented.

    The problem can be seen as an analogous to the \emph{Fair Bet Casino} one.
        Therefore, to solve the problem we consider the so called \emph{\gls{hmm}}.

    Formally, a HMM is defined by the 4-tuple \((Q, \Sigma, A, E)\) where \(Q\)
        is a set of states with some unknown probabily distribution, \(\Sigma\)
        is the output alphabet, \(A \subseteq Q \times Q\) stores the probabily
        of going from state \(i\) to state \(j, i, j \in Q\) and \(E \subset Q \times E\)
        stores the probabily of emmitting a symbol \(\sigma\) while in state \(k\).

    \subsection{The decoding problem}
    \subfile{../sub/5.1 - The decoding problem}
\end{document}
