\documentclass{subfiles}
\begin{document}
    Before we present Fibonacci codes, 
        let us recall that Fibonacci numbers are defined by the recurrence
        \[
            F_{n} = \begin{cases}
                0, \quad n = 0; \\
                1, \quad n = 1; \\ 
                F_{n - 1} + F_{n - 2}, n \ge 2
            \end{cases}
        \]
    Let us also introduce a foundamental result for the existence of Fibonacci codes.
    \begin{theorem*}[Zackendorf]
        Any given positive integer can be uniquely wrote
            as the sum of non consecutive Fibonacci numbers. 
    \end{theorem*}

    Using the above theorem, we can constuct a code that efficiently encodes integers.
    Let \(n\) be a positive integer, we encode it as follow:
        \begin{enumerate}
            \item Find the greatest Fibonacci number lower or equal to \(n\).
            \item Say it was the i-th Fibonacci number, subtract it from \(n\)
                and keep trace of the remainder. Set the (i-1)-th bit to 1.
                (The left most bit has index 0).
            \item Reapet the above steps by substituting \(n\) with the remainder,
                until it's 0.
            \item Append an addiotional 1 to the right of the codeword.
        \end{enumerate}

        \begin{example*}
            Let \(n = 73\), it's easy to se that it can be wrote as
                \(F_{10} + F_{7} + F_{5}\). From which by the above steps,
                we get that \(n\) is encode as \(0001010011\).
        \end{example*}

    For what regards the decoding, 
        we simply remove the additional 1 at the right of the codeword,
        and starting from the left side we replace the i-th 1 with the 
        (i + 1)-th Fibonacci number, then sum everything.
        \begin{example*}
            Let us decode \(0001010011\). 
                As said, we remove the extra bit at the right, 
                from which we get \(000101001\).
            Then, we replace the i-th 1 with the (i + 1)-th Fibonacci number. 
            Thus, we get \( F_{4 + 1} + F_{6 + 1} + F_{9 + 1}\),
                which basically is \(F_{5} + F_{7} + F_{10} = 73\).
        \end{example*}
\end{document}
