\documentclass{subfiles}
\begin{document}
    Since Propositional Logic is, before everything, a logic it has rules 
        which allows to deduce the truth of a sentence given another.
        Such rule system is a system of formal proof. In particular, we are 
        interested in the relationship between the notion of formal proof and that 
        of consequence; that is, we want our rule system to be \emph{sound}.

        \begin{definition*}[soundness and completeness]
            Let \(P = \set{p_{1}, p_{2}, \ldots}\) be a set ot atomic propositions,
            and let \(\Fam(P)\). We say that our rule system is sound 
            if every formula derived from \(\Fam\) is a consequence of 
            some proposition in \(\Fam\). That is, if
            \[
                \set*{G}[\Fam \vdash G][:] \subseteq \set*{G}[\Fam \models G][:].
            \]

            If it holds that
            \[
                \set*{G}[\Fam \models G][:] \subseteq \set*{G}[\Fam \vdash G][:],
            \]
            we say that our rule system is complete.
        \end{definition*}

    It can be shown that Propositional Logic is both sound and complete.
\end{document}
