\documentclass{subfiles}
\begin{document}
    A trickier implementation of the ElGamal algorithm makes use 
        of elliptic curves. Recall that the ``sum'' of two points \(P\) and 
        \(Q\) of some elliptic curve \(\C*\) is not trivial. 

    Back to the algorithm: a TTP fixes a elliptic curve \(\C*\) over the some field \(\gf{p}\). 
        Analogously to the previous case, an element \(\gamma \in \C*\) is fixed 
        and the quantitie \(\alpha = a\gamma\) is computed and published over the channel.
        The remainder of the algorithm is similar to that previously discussed; 
        the main difference is that multiplications are replaced by additions.

    \begin{remark*}
        Why do we prefer elliptic curves then? The answer is simple: the computation of 
            \(n\gamma\) is expensive.
    \end{remark*}
\end{document}
