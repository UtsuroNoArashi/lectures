\documentclass{subfiles}
\begin{document}
    If \(\mathcal{C}\) is represented by means of Cartesian equations,
        this means that any \(\overline{c} \in \mathcal{C}\) satisfies
        a system of equations of the form 
        \[\begin{cases}
            \alpha_{11}a_{1} + \alpha_{12}a_{2} + \ldots + \alpha_{1n}a_{n} = 0 \\
            \alpha_{21}a_{1} + \alpha_{22}a_{2} + \ldots + \alpha_{2n}a_{n} = 0 \\
            \qquad \vdots \qquad \vdots \qquad \vdots \qquad \vdots \qquad \vdots \\
            \alpha_{(n - k)1}a_{1} + \alpha_{(n - k)2}a_{2} + \ldots + \alpha_{(n - k)n}a_{n} = 0 \\
        \end{cases}\]
        with \(\alpha_{ij} \in A\).

    By elementary linear algebra, we can symply consider the matrix \(H\) of coefficient 
        of the linear equations. We call \(H\) parity matrix. Hence,
        any \(\overline{c} \in \mathcal{C}\) is such that \(\overline{c}H^{T} = \overline{0}\).
\end{document}
