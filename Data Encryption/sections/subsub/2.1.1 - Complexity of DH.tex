\documentclass{subfiles}
\begin{document}
    In our analisis of Diffie-Hellman complexity we distinguish two case: 
        the complexity for A and B, and the complexity for E.

    Consider A, they have to compute \(\alpha = \gamma^{a}\) 
        and \(k = \beta^{a}\) which requires \bigO{n^{3}} time.
        Similarly, B has to compute \(\beta = \gamma^{b}\) and \(k = \alpha^{b}\),
        again in \bigO{n^{3}} time. 

    If we consider E, they have to compute either \(a\) or \(b\).
        Then, they can either try by bruteforce, which requires \bigO{e^{n} \cdot n^{2}} time;
            or if they choose a randomised algorithm, it would take \bigO{\sqrt{e^{n} \cdot n^{2}}} time.

    \begin{remark*}
        Recall that with our notation, \(n = \floor{\log p} + 1\). 
            In addition, we can assume that \(k \simeq p\). 
    \end{remark*}
\end{document}
