\documentclass{subfiles}
\begin{document}
    Consider the following: can two parallel line meet? 
        Any reader with little to no background in geometry will 
        provide a negative answer to the question. Which, to be clear,
        is the most obvious answer. 

    This said, let us consider a real life example:
        take for instance the rails of a railway, though this are indeed parallel 
        if we look further enougth these merge at a point. More precisely,
        their projection meets at a point.
        This is the idea behind projective geometry.

    Formally speaking we have the following: let \(\F\) be a field, we call 
        projective plane of dimension \(n\) the set 
        \[
            \Ps_{n}(\F) = \set*{P = \gen{\vec{v}}}[\vec{0} \ne \vec{v} \in \F^{n + 1}],
        \]
        where \(\vec{v} = [x_{0}, \ldots, x_{n}]\) and \(\gen{\vec{v}} = \set{\lambda \vec{v}}[\lambda \in \F]\).

    Throughout this section we will consider \(\Ps_{2}\). Observe that
    \[
        \Ps_{2} = \set{p = \gen{\vec{v}}}[\vec{v} = [x_{0}, y_{0}, z_{0}], z_{0} \ne 0] \cup 
            \set{q = \gen{\vec{v}}}[\vec{v} = [x_{0}, y_{0}, z_{0}], z_{0} = 0].
    \]
    Given a point \(p \in \Ps_{n}\), and in particular in \(\Ps_{2}\), 
        we say that \(p\) is a proper point (or affine point) 
        if \(p = \gen{[x_{0}, y_{0}, z_{0}]}, z_{0} \ne 0\);
        we say that \(p\) is an improper point (or point at infinity) 
        if \(p = \gen{[x_{0}, y_{0}, z_{0}]}, z_{0} = 0\).
        Here by \(0\) we refer to the zero element of the field.
    
    \subsection{Elliptic curves}
    \subfile{../sub/4.1 - Elliptic curves}
\end{document}
