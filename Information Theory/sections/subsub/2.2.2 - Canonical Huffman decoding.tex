\documentclass{subfiles}
\begin{document}
    Let \(w\) be a text encoded using the Canonical Huffman encoding.
        Let \(l = 1\) be a counter and \(v\) the first bit read from \(w\).
        To retrieve \(w\)'s content we check whether \(v <\) \lstinline{fc[l]};
        if so, we shift \(v\) by one to the left and scan the next bit and increment \(l\) by one,
        if not we return the symbol \lstinline{symb[l,  v - fc[l]]}.

    \emph{Figure \ref{Fig:2}.\subref{Fig:2.b}} shows a C implementation of the procedure just described:
        here we assume \lstinline{next_bit} to be some existing primitive.
    We note that the procedure shown in (\subref{Fig:2.a}) builds \lstinline{fc}
        using a reverse approach; this, in general, produces a \lstinline{fc} array that 
        differs from one built using the forward approach.
        \clearpage
        \subfile{../../extra/TikZ/Figure 2 - Canonical Huffman snippets}
    \begin{example*}
        Let \(\C*\) be the Huffman code obtained from the tree in \emph{Figure \ref{Fig:1.b}}.
        \begin{enumerate}
            \item Let \(l_{i}\) denote the length of the \(i\)-th codeword:
                \(l_{0} = l_{1} = l_{2} = 2, \; l_{3} = l_{4} = 3\).

            \item Computing \lstinline{num} yields:
                \lstinline{num[1] = 0}, 
                \lstinline{num[2] = 3}, and 
                \lstinline{num[3] = 2}.

            \item Computing \lstinline{symb} gives:
                \lstinline{symb[1] = null}, 
                \lstinline{symb[2, 0] = a}, 
                \lstinline{symb[2, 1] = c}, 
                \lstinline{symb[2, 2] = b}, 
                \lstinline{symb[3, 0] = d}, and 
                \lstinline{symb[3, 1] = e}.
                For convenience, a second index is used to navigate within a list when needed.

            \item Computing \lstinline{fc} yields:
                \lstinline{fc[1] = 2} 
                (used in the decoding process),
                \lstinline{fc[2] = 1}, and
                \lstinline{fc[3] = 0}.

            \item Observe that \lstinline{symb[1] = null}, 
                indicating that there are no codewords to assign.
                For \lstinline{symb[2]}, assign \(01\) to \(a\), \(10\) to \(c\), and \(11\) to \(b\). 
                For \lstinline{symb[3]}, assign \(000\) to \(d\) and \(001\) to \(e\).
        \end{enumerate}
            
            Now let \(w = 01 10 000 001 10 11\) be a text encoded using the Canonical Huffman encoding,
                and let \lstinline{symb} and \lstinline{fc} be those just computed.
            Applying the procedure in (\subref{Fig:2.b}), one can verify that \(w = acdecb\).
    \end{example*}
\end{document}
