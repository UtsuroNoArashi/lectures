\documentclass{subfiles}
\begin{document}
    The \gls{bwt}, introduced by \emph{M. Burrows and D. Wheeler} in 
    \cite{burrows1994}, is a simple yet efficient tool to enhance the 
    compressibility of a given string.

    Let \(\omega\) be the string to be compressed and thus preprocessed 
    using the BWT. The first step of the transform is to compute all 
    lexicographically\footnotemark, and finally, the last column\footnotemark is 
    returned together with the index (starting from 0) of the original string.
    \begin{example*}
        Let \(\omega = aleph\). Its cyclic rotations, already sorted, are
        \[
            aleph \;, ephal \;, halep \;, lepha \;, phale
        \]
        Thus, we return the pair \((hlpae, 0)\).
    \end{example*}

    We now highlight some important properties of the BWT.
    \begin{enumerate}
        \item \(\forall i \ne I\), where \(I\) is the index of the original 
              string, \(F[i]\) follows \(L[i]\) in \(\omega\).
        \item \(F[I]\) is the first symbol of \(\omega\).
        \item For any character \(x\), the \(i\)-th occurrence of \(x\) in \(F\)
              corresponds to the \(i\)-th occurrence of \(x\) in \(L\).
    \end{enumerate}

    \footnotetext{For the sake of this discussion, assume that the 
    lexicographic order is the usual alphabetical one. This is, in general, 
    not true.}
    \footnotetext{By last column we mean the concatenation of the last symbol of each rotation.},
    cyclic rotations of \(\omega\). Next, these rotations are sorted 

    \subsubsection{Inverse of the BWT}
    \subfile{../subsub/5.1.1 - Inverse BWT}
\end{document}

