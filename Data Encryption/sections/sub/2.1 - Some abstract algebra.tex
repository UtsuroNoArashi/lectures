\documentclass{subfiles}
\begin{document}
    In what follows, we assume the reader to have a basic understanding of number sets 
    and equivalence relations.

    \begin{definition*}[Divisibility]
        Let \(a, b \in \Z\). We say that \(a\) divides \(b\), and write \(\divides{a}{b}\),
        if 
        \[
            \exists q \in \Z \text{ s.t. } b = qa.
        \]
        Otherwise we say that \(a\) does not divide \(b\), and write \(\ndivides{a}{b}\).
    \end{definition*}

    Divisibility allows us to define two important concepts: LCM and GCD.
    \begin{definition*}[LCM]
        Let \(a, b \in \Z\). Let \(m \in \Z\). We say that \(m\) is the LCM for 
            \(a \text{ and } b\) if, the following holds:
            \begin{enumerate}
                \item \(\divides{a}{m}\) and \(\divides{b}{m}\), and 
                \item If \(\exists m' \in \Z\) such that \(\divides{a}{m'}\) and 
                    \(\divides{b}{m'}\), then \(\divides{m}{m'}\).
            \end{enumerate}
    \end{definition*}

    \begin{definition*}[GCD]
        Let \(a, b \in \Z\). Let \(d \in \Z\). We say that \(d\) is the GCD for 
            \(a \text{ and } b\) if, the following holds:
            \begin{enumerate}
                \item \(\divides{d}{a}\) and \(\divides{d}{b}\), and 
                \item If \(\exists d' \in \Z\) such that \(\divides{d'}{a}\) and 
                    \(\divides{d'}{b}\), then \(\divides{d'}{d}\).
            \end{enumerate}
    \end{definition*}

    The computation of the GCD is done by the euclidean algorithm;
        we present an efficient inpementation of it in \emph{Section \ref{Sec:4}}.
        \todo{Add label in section 4.}

    \begin{definition*}[Congruence modulo \(n\)]
        Let \(a, b \in \Z\), and let \(n \in \Z\) be a fixed integer. 
        We say that \(a\) is congruent \(b\) modulo \(n\), and write 
        \(a \equiv b \mod{n}\) if and only if 
        \[
            \exists k \in \Z \text{ s.t. } a = kn + b.
        \]
    \end{definition*}

    \subsubsection{A brief introduction to groups}
    \subfile{../subsub/2.1.1 - Groups}

    \subsubsection{A brief introduction to rings}
    \subfile{../subsub/2.1.2 - Rings.tex}
\end{document}
