\documentclass{subfiles}
\begin{document}
    The first to solve the problem of creating optimal code was \emph{D.A. Huffman},
       who in \cite{Huffman} provides a method to create compact codes.

        Before we actually analyze such method, 
            it's of intrest to point out some properties that optimal codes 
            must satisfy.
        Let \(S\) be a source with some probability distribution,
        let's also assume that \(p_{1} \ge p_{2} \ge \ldots \ge p_{n}\).
        Let \(C\) be a compact prefix code for \(S\) such that \(c_{i}\) is 
            the codeword associated to \(p_{i}\). Then:
        \begin{enumerate}
            \item To reduce the expected code length, 
                the shortes codewords are associated to the symbols with the higtest
                probability. This means that 
                \[
                   p_{i} \ge p_{j} \implies \Abs{c_{i}} \le \Abs{c_{j}}
                \]
                If this was not the case, 
                swapping the codewords would result in a code with lower \gls{acl}.

            \item The least probable symbols have codewords with the same length.
            \item The longest codewords differ only by the last symbol.
        \end{enumerate}

        \subsection{The algorithm}
        \subfile{../sub/2.1 - HE the algorithm}

        \subsection{Canonical Huffman}
        \subfile{../sub/2.2 - Canonical Huffman}
\end{document}
