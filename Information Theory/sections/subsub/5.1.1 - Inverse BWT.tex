\documentclass{subfiles}
\begin{document}
    Since the BWT is a transform, it is desirable for it to be reversible. 
    Indeed, the BWT is reversible and admits two distinct but equivalent 
    inverse transforms: the FL-mapping and the LF-mapping.

    We illustrate these inverse transforms using the BWT of the previous 
    example as input.

    \paragraph*{FL-mapping}
    Let \(L = BWT(aleph)\) and \(I = 0\) be the index of the original 
    string. We construct \(F\) by lexicographically sorting \(L\). 
    Define the permutation that maps the symbols in \(F\) to those in \(L\) 
    as
    \[
        \tau = \begin{pmatrix}
            0 & 1 & 2 & 3 & 4 \\ 
            3 & 4 & 0 & 1 & 2
        \end{pmatrix}\text{.}
    \]
    We recover \(\omega\) by computing \(\omega[i] = F[\tau^{i}[I]]\) for 
    \(i = 0, 1, \ldots, 4\). Concretely, 
    \(\omega[0] = F[I] = a\), \(\omega[1] = F[\tau[0]] = F[3] = l\), and 
    so forth.

    \paragraph*{LF-mapping}
    Let \(L = BWT(aleph)\) and \(I = 0\) be the index of the original string. 
    We construct \(F\) by lexicographically sorting \(L\). Define the 
    permutation that maps the symbols in \(L\) to those in \(F\) as
    \[
        \sigma = \begin{pmatrix}
            3 & 4 & 0 & 1 & 2 \\
            0 & 1 & 2 & 3 & 4
        \end{pmatrix}\text{.}
    \]
    We recover \(\omega\) by computing \(\omega[n - 1 - i] = L[\sigma^{i}[I]]\) 
    for \(i = 0, 1, \ldots, 4\). Concretely, 
    \(\omega[4] = L[I] = h\), \(\omega[3] = L[\sigma[0]] = L[2] = p\), and 
    so forth.
\end{document}
