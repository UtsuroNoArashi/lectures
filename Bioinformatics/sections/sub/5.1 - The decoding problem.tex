\documentclass{subfiles}
\begin{document}
    Given a sequence of observations \(x\), i.e., a sequence of output symbols,
        is it possible to find a path \(\pi\), that is, a sequence of hidden states,
        such that \(\Prob{x}[\pi]\) is maximized?
    It turns out that there exists an algorithm that does exactly so: the Viterbi algorithm.
    
    Briefly, we compute \(s_{l, i + 1}\), which represents the probability of reaching state \(l\)
    and produce the prefix \(x_{1} \cdots x_{i + 1}\), as 
    \[
        \max\limits_{k \in Q} \set{s_{k, i} \cdot a_{kl} \cdot e_{l}(x_{i + 1})}
    \]
    where \(s_{start, 0} = 1\) and \(s_{k, 0} = 0, \forall k \ne start\).
    
    Since the values obtained by the algorithm are very small, in most cases we consider 
        the logarithm of such quantities.

    As an alternative approach one can conpute the best path as follow:
        consider the forward probability \(f_{k, i}\) of producing \(x_{1} \cdots x_{i}\)
        reaching state \(k\), defined by the recurrence
        \[
            f_{k, i} = e_{k}(x_{i}) \sum\limits_{l \in Q} f_{l, i - 1} a_{lk}
        \]
        and the backwards probability \(b_{k, i}\) of emitting \(x_{i + 1} \cdots x_{n}\)
        starting from state \(k\), in turns defined as 
        \[
            b_{k, i} =\sum\limits_{l \in Q}e_{l}(x_{i + 1}) b_{l, i + 1} a_{kl}.
        \]
    At any moment \(i\) we have that 
        \[
            \Prob{\pi_{i} = k, x} = \frac{\Prob{x, \pi_{i} = k}}{\Prob{x}} 
                = \frac{f_{k, i} b_{k, i}}{\Prob{x}}.
        \]
\end{document}
