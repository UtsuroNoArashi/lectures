\documentclass{subfiles}
\begin{document} 
    Let \(\omega\) be a string to encode, 
    produced by some source \(S\).
    To encode such string we have to:
        \begin{enumerate}
            \item Scan \(\omega\) to get the probabilities of each symbol.
                Unless these are known a priori.
            \item Sort the symbols non-increasingly according to their probabilities.
            \item Build a (binary) tree where the leaves are the source symbols,
                and each node is the sum of the (two) lowest probabilities.
                Repeat the process until a root is formed.
            \item Build the code by traversing the tree from the root to a leaf,
                and assigning a 0 to the left paths, a 1 to the right one.
        \end{enumerate}

        \begin{example*}
            To ease things out, the example skips the scanning step.

            Let \(S = \set{a, b, c, d, e}\) whose probabilities are
            \(\set*{\tfrac{1}{3}, \tfrac{1}{6}, \tfrac{1}{3},
            \tfrac{1}{12}, \tfrac{1}{12}}\), respectively. 
            Applying the steps above we have the following.
            \begin{description}
                \item [\textbf{Step 1.}] 
                    Sorting the symbols according to the probabilities,
                    we get the following order: \(a, c, b, d, e\).

                \item [\textbf{Step 2.}] 
                    To build the tree let's consider the 
                    two least probable symbols, in this case \(d \text{ and } e\).
                    The sum of their probabilities will define a new node \(x\) in the
                    tree (the \Csquare{rpIris} one in the figures below).
                    The next node \(y\) (the \Csquare{rpFoam} one) is given by \(x\) 
                    and the third least probable symbol \(b\).
                    At this point we can make two choices to define the node \(z\):
                    we can either merge \(a \text{ and } c\) or merge \(c \text{ and } y\).
                    Both will build a optimal code. 
                    
                    For the sake of this example, let's merge \(c \text{ and } y\),
                    the other choice is shown in \emph{Figure \ref{Fig:2.b}}.
                    Lastly, we merge the last two symbols, 
                    leading to the tree in \emph{Figure \ref{Fig:2.a}}.
                    \subfile{../../extra/TikZ/Figure 1 - Example of Huffman encoding}

                \item [\textbf{Step 3.}] Traversing the tree we get 
                    \(C = \set{0, 10, 110, 1110, 1111}\) for tree in 
                    \emph{Figure \ref{Fig:2.a}} or \(C = \set{00, 01, 10, 110, 111}\)
                    for the one in \emph{Figure \ref{Fig:2.b}}.
            \end{description}   
        \end{example*}

        From the steps above, it's obvious that the slowest part of the encoding
        is given by the neccessity to scan the text twice. 
\end{document}
