\documentclass{subfiles}
\begin{document}\label{Sec:4}
    What occurs when the source to be encoded employs positive integers as 
    symbols? How can one determine a universal representation? More precisely, 
    how can we construct a code that is prefix-free and whose ACL is 
    \bigO{\ln x} for any \(x\)?

    Is it possible to employ the previously discussed codes, or is it necessary 
    to develop new ones? The answer depends on the context. For example, if 
    the distribution of the integers is known and the range is relatively small, 
    Huffman coding remains applicable. Otherwise, alternative coding schemes, 
    which will be discussed in the subsequent sections, are required.

    \begin{remark*}
        The restriction to positive integers can be relaxed by mapping any 
        positive integer \(x\) to \(2x + 1\) and any negative integer \(y\) 
        to \(-2y\).
    \end{remark*}

    \subsection{Simple approaches}
    \subfile{../sub/4.1 - IE simple approaches}

    \subsection{Elias codes}
    \subfile{../sub/4.2 - Elias codes}

    \subsection{Fibonacci code}
    \subfile{../sub/4.3 - Fibonacci code}
    \subfile{../exercise/Exercises - Integer Encoding}
    \cleardoublepage
\end{document}
